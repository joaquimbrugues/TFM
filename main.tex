\documentclass[a4paper,11pt]{book}
\usepackage[utf8]{inputenc}
\usepackage[english]{babel}
\usepackage{import}
\usepackage{tikz}
\usepackage{float}
\usepackage{pgfplots}
\usetikzlibrary{arrows.meta}
\usetikzlibrary{decorations.markings}
\usepackage{styles/TFG}
\usepackage{styles/TFM}
\usepackage[hidelinks,pdfauthor={Joaquim Brugues Mora},pdftitle={Master thesis Floer Homology},pdftex]{hyperref}

\begin{document}

%----------------- Title page --------------------%
\begin{titlepage}
	\centering
	{\scshape\LARGE Universitat Politècnica de Catalunya - BarcelonaTech\par}
	{\scshape\LARGE Facultat de Matemàtiques i Estadística, \par}
	\vspace{1cm}
	{\scshape\Large Master Thesis\par}
	\vspace{1.5cm}
	{\huge\bfseries Morse Theory and Floer Homology\par}
	\vspace{2cm}
	{\Large\itshape Joaquim Brugués Mora\par}
	\vfill
	advisors\par
	{\Large\itshape Eva Miranda Galceran \\ Cédric Oms \par}

	\vfill

% Bottom of the page
	{\large \today\par}
\end{titlepage}

%----------------- Summary --------------------%

\chapter*{Summary}

%TODO: Here we should put the summary and keywords for this project.

%----------------- Table of contents --------------------%
\tableofcontents

\mainmatter

\chapter*{Introduction}
\fancyhead[LE]{Introduction}
\addcontentsline{toc}{chapter}{Introduction}
\import{contents/}{introduction.tex}

%Chapter 1, on Morse theory
\chapter{Fundamentals on Morse homology}
\fancyhead[LE]{Morse Theory}
In this chapter we explain the basic constructions on Morse theory. Our aim is not to achieve a deep understanding of it, but to get familiar with the way to prove the results and the kind of tools that are used. We do so because Floer homology is constructed using ideas analogous to the ones presented here (in a more complex setting), so it is convenient to be familiar with Morse theory before getting started with Floer homology.

This chapter is based on the introduction of Morse homology that can be found in the first chapters of \cite{audin2014morse}. There is also an excellent presentation in \cite{milnor1963morse}, which has also been used.

\fancyhead[RO]{Morse functions}
\import{contents/}{morse1.tex}

\fancyhead[RO]{Applications to topology}
 \import{contents/}{morse4.tex}

\fancyhead[RO]{The Morse complex}
 \import{contents/}{morse2.tex}

\fancyhead[RO]{Pseudogradients and the Smale condition}
 \import{contents/}{morse3.tex}

\fancyhead[RO]{The differential on the Morse complex}
 \import{contents/}{morse5.tex}

\fancyhead[RO]{Examples of the Morse complex}
\import{contents/}{morse8.tex}

\fancyhead[RO]{The Morse homology is well defined}
 \import{contents/}{morse6.tex}

%Chapter 2, symplectic geometry
 \chapter{Symplectic manifolds and the Arnold conjecture}
\fancyhead[LE]{Symplectic manifolds and the Arnold conjecture}
In this chapter we are going to present the context in which Floer homology arises, which is the one of symplectic manifolds. Roughly speaking, the structure of a symplectic manifold emerges when trying to generalise the dynamics of Hamiltonian systems from $\R^{2n}$ to a $2n$-dimensional manifold. A thorough description of the theory can be found in \cite{da2001lectures}, and we will assume that the reader is familiar with the basic concepts.

We will also present the first motivation for the development of Floer theory, which is the Arnold conjecture on fixed points of symplectomorphisms.

\fancyhead[RO]{A toolkit of symplectic geometry}
\import{contents/}{symplectic1.tex}

\fancyhead[RO]{The Arnold conjecture}
\import{contents/}{symplectic2.tex}

\fancyhead[RO]{The conditions on the manifold}
\import{contents/}{symplectic3.tex}

 % Chapter 3, the Floer equation
 \chapter{The Floer equation}
 \fancyhead[LE]{The Floer equation}
 In this chapter we are going to define the first constructions needed to define the Floer complex, namely the loop space, the action functional and the flow of the Floer equation. In every step we are going to refer to the Morse homology and the steps proposed at Section \ref{section:morse_complex}.

 \fancyhead[RO]{The space of contractible loops}
 \import{contents/}{floereq1.tex}

 \fancyhead[RO]{The action functional}
 \import{contents/}{floereq2.tex}

 \fancyhead[RO]{The Floer equation}
 \import{contents/}{floereq3.tex}

 \fancyhead[RO]{$\mathcal{M}$ is compact}
 \import{contents/}{floereq4.tex}

 \fancyhead[RO]{Limit endpoints of elements of $\mathcal{M}$}
 \import{contents/}{floereq5.tex}

%Chapter 4, the Floer homology
\chapter{The Floer complex}
\fancyhead[LE,RO]{The Floer complex}
\import{contents/}{floercomplex.tex}

%Conclusions
\chapter*{Conclusions}
\fancyhead[LE,RO]{Conclusions}
\import{contents/}{conclusion.tex}

%----------------- Appendices ---------------------%
\begin{appendices}
\fancyhead[LE,RO]{Appendices}
\chapter{Additional results on Morse theory}
\import{contents/}{morse7.tex}

\chapter{The Conley-Zehnder index} \label{chapter:conley_zehnder}

In this chapter we will give a basic definition of the Conley-Zehnder index of a path of symplectic matrices. In the first section we will give a justification of why we are interested in such a map and how do the principal properties we ask from it arise. In the second section we will study the topology of the Lie group $\sympl(\R^{2n},\Omega_0)$, which will provide us with the tools needed to define the rotation map. Finally, we will use all these tools to define the Conley-Zehnder index and to prove its most essential properties.

In this chapter we follow the scheme of \cite{gutt2012conley}.

\import{contents/}{conleyzehnder1.tex}
\import{contents/}{conleyzehnder2.tex}
\import{contents/}{conleyzehnder3.tex}
\end{appendices}

%----------------- Bibliography --------------------%
%Here goes the bibliography of the project
\bibliographystyle{alpha}
\bibliography{references}
\addcontentsline{toc}{chapter}{Bibliography}
\end{document}
