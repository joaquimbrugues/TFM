\documentclass[a4paper,11pt]{book}
\usepackage[utf8]{inputenc}
\usepackage[english]{babel}
\usepackage{import}
\usepackage{styles/TFG}
\usepackage[pdfauthor={Joaquim Brugues Mora},pdftitle={Master thesis Floer Homology},pdftex]{hyperref}

%Headers and footers
\fancyfoot[LE, RO] {\thepage}
\fancyfoot[C] { }

%Define Therorems, propositions, corollaries...
\newtheoremstyle{indented}{\topsep}{\topsep}{\normalfont}{12pt}{\bfseries}{}{5pt plus 1pt minus 1pt}{}
\theoremstyle{indented}
\newtheorem{theo}{Theorem}[chapter]
\newtheorem{prop}[theo]{Proposition}
\newtheorem{lema}[theo]{Lemma}
\newtheorem{coro}[theo]{Corollary}
\newtheorem{deff}{Definition}[chapter]
\newtheorem{rmrk}{Remark}[chapter]
\newtheorem{exmpl}{Example}[chapter]

%Some special commands
\newcommand{\dd}{\operatorname d}
\newcommand{\straightT}{\operatorname T}
\newcommand{\straightH}{\operatorname H}
\newcommand{\crit}{\text{Crit}}
\newcommand{\grad}{\text{grad}}
\newcommand{\indx}{\text{Ind}}
\newcommand{\Id}{\text{Id}}

\begin{document}

%----------------- Title page --------------------%
\begin{titlepage}
	\centering
	{\scshape\LARGE Universitat Politècnica de Catalunya - BarcelonaTech\par}
	{\scshape\LARGE Facultat de Matemàtiques i Estadística, \par}
	\vspace{1cm}
	{\scshape\Large Master Thesis\par}
	\vspace{1.5cm}
	{\huge\bfseries Floer Homology\par}
	\vspace{2cm}
	{\Large\itshape Joaquim Brugués Mora\par}
	\vfill
	advisor\par
	{\Large\itshape Eva Miranda Galceran \\ Cédric Oms \par}

	\vfill

% Bottom of the page
	{\large \today\par}
\end{titlepage}

%----------------- Summary --------------------%

\chapter*{Summary}

%TODO: Here we should put the summary and keywords for this project.

%----------------- Table of contents --------------------%
\tableofcontents

\mainmatter

%Chapter 1, on Morse theory
\chapter{Fundamentals on Morse homology}
\fancyhead[LE]{Morse Theory}
In this chapter we explain the basic constructions on Morse theory. Our aim is not to get a deep understanding of the theory, but to get familiar with the way how the results are proved and which kind of tools are used. We do so because Floer homology will use concepts that are analogous to the ones presented here (but, of course, in a more complex setting), so it is convenient to be familiar with Morse homology before jumping to the Floer homology.

\fancyhead[RO]{Morse functions}
\import{contents/}{morse1.tex}

\fancyhead[RO]{The Morse complex}
 \import{contents/}{morse2.tex}

\fancyhead[RO]{Topology nearby a critical point}
 %\import{contents/}{morse3.tex}

\fancyhead[RO]{Pseudogradients and the Smale condition}
 \import{contents/}{morse4.tex}

\fancyhead[RO]{The differential on the Morse complex}
 %\import{contents/}{morse5.tex}

\fancyhead[RO]{The Morse homology is well defined}
 \import{contents/}{morse6.tex}

%----------------- Bibliography --------------------%
%Here goes the bibliography of the project
\end{document}
