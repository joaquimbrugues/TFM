\section{Limit endpoints of elements of $\mathcal{M}$}

We devote this section to prove what we already hinted in the remark \ref{remark:floer_limits}: the elements of $\mathcal{M}$ connect the critical points of the action functional. The precise statement of the theorem is the following:

\begin{theo} \label{theo:floer_endpoints}
If all the trajectories of $X_t$ are non-degenerate, and under the assumption of asphericallity, then for all $u \in \mathcal{M}$ there are $x, y \in \mathcal{L}M$ critical points of $\mathcal{A}_H$ such that
\[\lim_{s\rightarrow - \infty} u(s,\cdot) = x \ , \ \ \ \lim_{s\rightarrow +\infty} u(s,\cdot) = y\]
in the $\mathcal{C}^{\infty}(\con{S}^1,M)$ sense, and
\[\lim_{s\rightarrow \pm\infty} \frac{\partial u}{\partial s} (s,t) = 0\]
uniformly in $t$.
\end{theo}

If we prove this theorem, we will have constructed the analogous as the pseudogradient adapted to a function in a manifold $M$ in Morse theory: a flow that connects critical points in a meaningful way. In the Morse case, this flow appears integrating the pseudogradient flow, but in the Floer case we needed to be a little more subtle and avoid the proper definition of the vector field, defining instead the equation that the flow must satisfy.

First of all, we need a basic lemma on what the nondegeneracy of $X_t$ implies for the critical points of the action functional:

\begin{lema} \label{lema:floer_finite}
Under the assumption of nondegeneracy, the number of critical points of $\mathcal{A}_H$ (or the number of periodic orbits of $X_t$) is finite.
\end{lema}

\begin{proof}
Let us consider the following submanifolds of $M\times M$:
\begin{itemize}
	\item The diagonal $\Delta = \{(x,x) \ | \ x \in M\}$.
	\item The graph of the flow $\varphi_{X_t}^t$ of $X_t$ at time $1$: $F = \left\{\left(x,\varphi_{X_t}^1(x)\right) \ | \ x \in M\right\}$.
\end{itemize}
The fact that the periodic orbits of $X_t$ are nondegenerate is equivalent to $\Delta \pitchfork F$. Moreover, $\dim(\Delta) = \dim(F) = 2n$ and $\dim(M\times M) = 4n$, so $\dim(\Delta \cap F) = 0$, so it is a closed submanifold of dimension $0$ of a compact manifold, so it is finite.
\end{proof}

In order to prove the theorem, we need to begin by proving a certain bound in the energy and the action functional, depending on the critical points of $\mathcal{A}_H$. Let us denote $u_s = u(s,\cdot) \in \mathcal{L}M$.

\begin{prop} \label{prop:floer_action_bound}
Let $u \in \mathcal{M}$. There exist $x,y \in \crit(\mathcal{A}_H)$ such that
\[\lim_{s \rightarrow -\infty} \mathcal{A}_H(u_s) = \mathcal{A}_H(x), \ \ \ \lim_{s \rightarrow +\infty} \mathcal{A}_H(u_s) = \mathcal{A}_H(y) .\]
\end{prop}

This has some immediate consequences:

\begin{coro}
If $\mathcal{M}$ is nonempty, then $\crit(\mathcal{A}_H)$ is nonempty.
\end{coro}

This does not even require a proof, as it is an immediate consequence of the proposition.

\begin{coro} \label{coro:floer_energy_bound}
There exist $C_1, C_2 > 0$ such that
\[|\mathcal{A}_H(u)| < C_1 , E(u) < C_2 \ \ \ \ \forall u \in \mathcal{M} .\]
\end{coro}

\begin{proof}
Because of the lemma \ref{lema:floer_finite}, we know that $\crit(\mathcal{A}_H)$ is finite. Therefore, $\mathcal{A}_H(\crit(\mathcal{A}_H))$ is a bounded set in $\R$. Moreover, the function $s \mapsto \mathcal{A}_H(u_s)$ is decreasing, because
\[\frac{d}{ds} \mathcal{A}_H(u_s) = \dd_{u_s} \mathcal{A}_H \cdot \frac{\partial u}{\partial s} = g\left( \grad_{u_s} \mathcal{A}_H, \frac{\partial u}{\partial s} \right) = - \left\|\frac{\partial u}{\partial s}\right\|^2 \leq 0 ,\]
(because $\grad_{u_s}\mathcal{A}_H = - \frac{\partial u}{\partial s}$).

Therefore,
\[\mathcal{A}_H(y) = \lim_{s \rightarrow +\infty} \mathcal{A}_H(u_s) \leq \mathcal{A}_H(u_s) \leq \lim_{s \rightarrow -\infty} \mathcal{A}_H(u_s) = \mathcal{A}_H(x) ,\]
from which follows the bound for $\mathcal{A}_H$. To deduce that the energy must be bounded in $\mathcal{M}$, it is enough to consider that
\[E(u) = \mathcal{A}_H(x) - \mathcal{A}_H(y)\]
because of the remark \ref{remark:floer_limits}, so the energy must also be bounded.
\end{proof}

\begin{proof} {\it (Proposition \ref{prop:floer_action_bound}):} Let us prove the case for $s \rightarrow +\infty$, as the case for $-\infty$ is analogous.

As we just showed, the function $s \mapsto \mathcal{A}_H(u_s)$ is decreasing. As it is also continuous, it is enough to prove that there exists some sequence $(s_k)_k$ with $s_k \xrightarrow[k \rightarrow \infty]{} +\infty$ and some $y \in \crit(\mathcal{A}_H)$ with
\[\lim_{k \rightarrow \infty} \mathcal{A}_H(u_{s_k}) = \mathcal{A}_H(y) .\]
We will prove this in 3 steps:
\begin{enumerate}
	\item There exists a $(s_k)_k$ tending to infinity such that $u_{s_k} \xrightarrow[k\rightarrow \infty]{} y$ in the $\mathcal{C}^0(\con{S}^1,M)$ topology for some $y \in \mathcal{C}^0(\con{S}^1,M)$.
	\item $y$ is of class $\mathcal{C}^{\infty}$, and $y \in \crit(\mathcal{A}_H)$.
	\item $\mathcal{A}_H(u_{s_k}) \xrightarrow[k \rightarrow \infty]{} \mathcal{A}_H(y)$.
\end{enumerate}

{\bf Step 1:} Let $u \in \mathcal{M}$. Its energy is finite, so
\[\int_{-\infty}^{+\infty} \left( \int_0^1 \left| \frac{\partial u}{\partial t} - X_t(u) \right|^2 dt \right) ds < \infty .\]
Therefore, there exists some sequence $(s_k)_k$ with $s_k \xrightarrow[k\rightarrow \infty]{} +\infty$ such that
\[\lim_{k\rightarrow \infty} \left\| \frac{\partial u}{\partial t}(s_k,\cdot) - X_t(u(s_k,\cdot)) \right\|_{L^2}^2 = 0 .\]
Let $u_k = u(s_k,\cdot)$. Taking an embedding of $M$ into $\R^m$, we can consider that
\[\lim_{k\rightarrow \infty} \left\| \dot{u}_k - X_t(u_k) \right\|_{L^2(\con{S}^1,\R^m)}^2 = 0 .\]
$M$ is compact, so $X_t$ is bounded, so there is some $R > 0$ such that $\displaystyle\sup_{p \in M, t \in \con{S}^1} \|X_t(p)\| < R$. Therefore, there is some $B > 0$ with $\|\dot{u}_k\|_{L^2} \leq B$ for all $k$. This implies that $(u_k)_k$ is equicontinuous:
\[\|u_k(t_1) - u_k(t_0)\|\]
\end{proof}
