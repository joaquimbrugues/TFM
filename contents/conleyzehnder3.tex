\section{The Conley-Zehnder index}

As we said in the introduction, the ultimate goal of this chapter is defining a map
\[\begin{array}{rccc} \mu : & \text{SP}(n) & \longrightarrow & \con{Z} \end{array}\]
that allows us to classify the paths of symplectic matrices according to their rotation map.

\begin{deff} The positive (resp. negative) components of $\sympl(\R^{2n},\Omega_0)$ are
\[\sympl(2n)^{\pm} = \{A \in \sympl(\R^{2n},\Omega_0) \ | \ \det(A - \text{Id}) \gtrless 0 \} .\]
\end{deff}

It can be seen that both $\sympl(2n)^+$ and $\sympl(2n)^-$ are path-connected. Therefore, it is possible for us to choose a "representative" element in each connected component,
\[W^+ = - \Id, \ \ \ W^- = \left( \begin{array}{c|c} \begin{array}{cc} 2 & 0 \\ 0 & \frac12 \end{array} & 0 \\ \hline 0 & - \Id \end{array} \right) .\]
We can easily compute $\rho$ in this matrices, as they are diagonal:
\[\rho(W^+) = \rho(- \Id) = \rho\left( \begin{pmatrix} -1 & 0 \\ 0 & -1 \end{pmatrix} \right) = (-1)^n ,\]
\[\rho(W^-) = \rho\left( \begin{pmatrix} 2 & 0 \\ 0 & \frac12 \end{pmatrix} \right) \rho\left(\begin{pmatrix}-1&0\\0&-1\end{pmatrix}\right)^{n-1} = (-1)^{n-1} .\]

The idea is to associate to each $\psi \in \text{SP}(n)$ an extended path $\widetilde{\psi} : [0,2] \rightarrow \sympl(\R^{2n},\Omega_0)$ such that $\left. \widetilde{\psi} \right|_{[1,2]} \subset \sympl(2n)^{\pm}$ and $\widetilde{\psi}(2) = W^{\pm}$. The main issue with this is that we can extend the same path $\psi$ in a lot of ways (even though the endpoint must always be $W^+$ or $W^-$ if $\psi(1) \in \sympl(2n)^+$ or $\psi(1) \in \sympl(2n)^-$, respectively). However, it can be proved that all the extensions of the same path are homotopically equivalent. To show this, we will prove the following theorem:

\begin{theo} \label{sppmcontractible}
Any continuous loop in $\sympl(2n)^+$ or $\sympl(2n)^-$ is contractible. 
\end{theo}

\begin{proof}
A loop $\gamma : \con{S}^1 \rightarrow \sympl(\R^{2n},\Omega_0)$ is contractible if, and only if, its image by $\rho$ is contractible. It is because $\rho$ induces an isomorphism $\rho_{\ast}$ between the first homotopy groups. Therefore, it suffices to see that $\rho \circ \gamma : \con{S}^1 \rightarrow \con{S}^1$ is contractible.

In the last section we introduced the lemma \ref{rotationformula} for $\rho$. Let us consider the map that sends each symplectic matrix $A$ to the set of its eigenvalues of first kind:
\[\begin{array}{ccc} \sympl(2n)^{\pm} & \longrightarrow & \Lambda_n = \quocient{\C^n}{\text{permutations of the elements}} \\ A & \longmapsto & \{\lambda_1,...,\lambda_n\} \end{array} ,\]
which is a continuous map. In other words, we can define the continuous maps $\Lambda_1,...,\Lambda_n : [0,1] \rightarrow \C$ such that $\Lambda_i(t)$ is the $i$-th eigenvalue of first kind of $\gamma(t)$. We want to define continuous maps $\alpha_i : [0,1] \rightarrow [0,2\pi]$ such that
\[e^{i \alpha_i(t)} = \frac{\Lambda_i(t)}{|\Lambda_i(t)|} ,\]
which is well defined whenever $\frac{\Lambda_i(t)}{|\Lambda_i(t)|} \neq 1$. In the case that $\frac{\Lambda_i(t)}{|\Lambda_i(t)|} = 1$, this means, that $\Lambda_i(t)$ is real and positive, we need to provide a consistent definition of $\alpha_i(t)$.

It can be seen that, if $\gamma(t) \in \sympl(2n)^+$, then the number of $\Lambda_i(t)$ that are real and positive is even, and, conversely, it is odd when $\gamma(t) \in \sympl(2n)^-$. Therefore, if the number is $2k$ (resp. $2k+1$), we can take $k$ (resp. $k+1$) of the $\alpha_i(t)$ to be $\alpha_i(t) = 2\pi$, and the remaining $k$ to be $\alpha_i(t) = 0$. It can be proved that this results in a set of continuous maps $\alpha_1,...,\alpha_n$.

Thus, if $\gamma : \con{S}^1 \rightarrow \sympl(2n)^{\pm}$, its image by each of the $\alpha_i$ is a loop $\alpha_i \circ \gamma : \con{S}^1 \rightarrow [0,2\pi]$ and therefore contractible. This means that
\[\rho(\gamma(t)) = e^{i \sum_{j=1}^n \alpha_j(t)}\]
is a contractible map, and, by our first observation in this proof, $\gamma$ is contractible.
\end{proof}

Therefore, if $\widetilde{\psi_1}$ and $\widetilde{\psi_2}$ are extensions of the same path $\psi \in \text{SP}(n)$, then $\widetilde{\psi_1} \simeq \widetilde{\psi_2}$.

Now we are ready to define the Conley-Zehnder index and to prove its most interesting properties.

\begin{deff}
Let $\psi \in \text{SP}(n)$. Its {\bf Conley-Zehnder index} is
\[\mu_{CZ}(\psi) = \text{deg}(\rho^2 \circ \widetilde{\psi}) ,\]
where $\rho^2$ means the product of $\rho$ with itself. The reason for this square is the need of having a map from $\con{S}^1$ to itself. We know that $\rho \circ \widetilde{\psi} : [0,2] \rightarrow \con{S}^1$ and that $(\rho \circ \widetilde{\psi})(2) = \pm 1$. As we need to have a periodic map to compute its degree, we square the result.
\end{deff}

\begin{rmrk}
The map is well defined, this means, it does not depend on the extension $\widetilde{\psi}$ chosen for $\psi$. This is a consequence of theorem \ref{sppmcontractible}, because any two such extensions are homotopic, so their compositions with $\rho^2$ are also homotopic, and therefore they have the same degree.
\end{rmrk}

\begin{prop} \label{conleyzehnderproperties} The Conley-Zehnder index has the following properties:
\begin{enumerate}
	\item {\bf Naturality:} For any path $\phi : [0,1] \rightarrow \sympl(\R^{2n},\Omega_0)$,
	\[\mu_{CZ}(\phi \psi \phi^{-1}) = \mu_{CZ}(\psi) .\]
	\item {\bf Homotopy:} If $\psi_0 \simeq \psi_1$, then $\mu_{CZ}(\psi_0) = \mu_{CZ}(\psi_1)$.
	\item {\bf Zero property:} If $\psi(s)$ has no eigenvalue on the circle for $s > 0$, then
	\[\mu_{CZ}(\psi) = 0 .\]
	\item {\bf Product property:} If
	\[\psi_1 \oplus \psi_2 = \begin{pmatrix} \psi_1 & 0 \\ 0 & \psi_2 \end{pmatrix} \in \sympl(\R^{2n},\Omega_0),\]
	then
	\[\mu_{CZ}(\psi_1 \oplus \psi_2) = \mu_{CZ}(\psi_1) + \mu_{CZ}(\psi_2) .\]
	\item {\bf Loop:} If $\phi : [0,1] \rightarrow \sympl(\R^{2n},\Omega_0)$ with $\phi(0) = \phi(1) = \Id$, then
	\[\mu_{CZ}(\phi \psi) = \mu_{CZ}(\psi) + 2 \text{deg}(\rho \circ \phi) .\]
	\item {\bf Signature:} If $S$ is a symmetric non-degenerate $2n \times 2n$ matrix with all eigenvalues satisfying $|\lambda_i| < 2\pi$ and $\psi(t) = \exp(J_0St)$, then
	\[\mu_{CZ}(\psi) = \frac12 \mathrm{Sign}(S),\]
	where $\mathrm{Sign}$ denotes the signature of $S$.
	\item {\bf Determinant:} If $\psi \in \text{SP}(n)$,
	\[(-1)^{n-\mu_{CZ}(\psi)} = \mathrm{sign} \left( \det(\Id - \psi(1)) \right) .\]
	\item {\bf Inverse:}
	\[\mu_{CZ}(\psi^{-1}) = \mu_{CZ}(\psi^T) = - \mu_{CZ}(\psi) .\]
\end{enumerate}
\end{prop}

\begin{proof} {\it (Naturality):} Let $\widetilde{\phi}$ be an extension of $\phi$ to $[0,2]$ such that $\phi(t) \in \sympl(\R^{2n},\Omega_0)$ for all $t$ and $\phi(2) = \Id$. Then, $\widetilde{\psi}' = \widetilde{\phi} \widetilde{\psi} \widetilde{\phi}^{-1}$ is a path connecting the identity with $W^{\pm}$, so we can compute
\[\text{deg}(\rho^2 \circ \widetilde{\psi}'),\]
and, by the naturality property of $\rho$, $\rho(\widetilde{\psi}'(t)) = \rho(\widetilde{\psi}(t))$ for all $t$, so
\[\mu_{CZ}(\phi\psi\phi^{-1}) = \text{deg}(\rho^2 \circ \widetilde{\psi}') = \text{deg}(\rho^2 \circ \widetilde{\psi}) = \mu_{CZ}(\psi) .\]
\end{proof}

\begin{proof} {\it (Homotopy):} We consider that $\psi_0 \simeq \psi_1$ in $\text{SP}(n)$, so there is a map $\psi : [0,1] \times [0,1] \rightarrow \sympl(\R^{2n},\Omega_0)$ with $\psi_s(1) \in \sympl(2n)^{\pm}$ for all $s \in [0,1]$.
Let $\widetilde{\psi_0}$ be an extension of $\psi_0$, and consider the extension $\widetilde{\psi_s}$ of $\psi_s$ defined by
\[\widetilde{\psi_s}(t) = \left\{ \begin{array}{lc} \psi_s(t) & 0 \leq t \leq 1 \\ \psi_{s(3-2t)}(1) & 1 \leq t \leq \frac32 \\ \widetilde{\psi_0}(2t-2) & \frac32 \leq t \leq 2 \end{array} \right. ,\]
which is continuous and defines a homotopy between $\widetilde{\psi_0}$ and $\widetilde{\psi_1}$. Therefore,
\[\mu_{CZ}(\psi_0) = \mu_{CZ}(\psi_1) .\]
\end{proof}

\begin{proof} {\it (Zero property):} By the normalization property and the continuity of $\rho$ we deduce that $\rho(\psi(t)) = 1 \ \forall t \in [0,1]$. Moreover, it is possible to construct an extension of $\psi$ such that its eigenvalues are outside of the circle for $t < 2$. Thus, we get that $\rho^2(\widetilde{\psi}(t)) = 1 \ \forall t \in [0,2]$. We conclude that $\mu_{CZ}(\psi) = \text{deg}(\rho^2 \circ \widetilde{\psi}) = 0$.
\end{proof}

\begin{proof} {\it (Product):} We will use the product property of $\rho$. Consider $\widetilde{\psi}_1$ and $\widetilde{\psi}_2$ the extensions of $\psi_1$ and $\psi_2$. If one (or both) of $\widetilde{\psi}_i(2)$ is equal to $W^-$, then $(\widetilde{\psi}_1 \oplus \widetilde{\psi}_2)(2) = W^{\pm}$. On the other hand, if $(\widetilde{\psi}_1 = W^+$ and $\widetilde{\psi}_2)(2) = W^+$ then $(\widetilde{\psi}_1 \oplus \widetilde{\psi}_2)(2)$ is homotopically equivalent to $W^+$, so we can find some path $\widetilde{\psi}' \simeq \widetilde{\psi}_1 \oplus \widetilde{\psi}_2$ with $\widetilde{\psi}'(2) = W^+$ and such that $\left. \widetilde{\psi}' \right|_{[0,1]} = \psi_1 \oplus \psi_2$. Therefore,
\[\mu_{CZ}(\psi_1 \oplus \psi_2) = \text{deg}(\rho^2 \circ \widetilde{\psi}') = \text{deg}(\rho^2 \circ (\widetilde{\psi}_1 \oplus \widetilde{\psi}_2)) =\]
\[= \text{deg}((\rho^2 \circ \widetilde{\psi}_1) \cdot (\rho^2 \circ \widetilde{\psi}_1)) = \text{deg}(\rho^2 \circ \psi_1) + \text{deg}(\rho^2 \circ \psi_2) = \mu_{CZ}(\psi_1) + \mu_{CZ}(\psi_2) .\]
\end{proof}

\begin{lema} \label{looplemma} Consider $\varphi, \psi : [0,T] \rightarrow \sympl(\R^{2n},\Omega_0)$ two paths of symplectic matrices with $\varphi(0) = \psi(0) = \Id$. Then, the product path $\psi\varphi$ is homotopic to the path
\[(\psi \diamond \varphi)(t) = \left\{ \begin{array}{lc} \varphi(2t) & t \leq \frac{T}2 \\ \psi\left( 2 \left( t - \frac{T}2 \right) \right) \varphi(T) & t \geq \frac{T}2 \end{array} \right. ,\]
so $\psi\varphi \simeq \psi \diamond \varphi$.
\end{lema}

\begin{proof} {\it (Lemma):} Consider the homotopy $\chi : [0,1] \times [0,T] \rightarrow \sympl(\R^2,\Omega_0)$ defined by
\[\chi_s(t) = \left\{ \begin{array}{lc} \varphi(2t) & t \leq \frac{sT}2 \\ \psi\left( \frac2{2-s} \left( t - \frac{sT}2 \right) \right) \varphi\left( sT + \frac{2(1-s)}{2-s} \left( t - \frac{sT}2 \right) \right) & t \geq \frac{sT}2 \end{array} \right. .\]
It is continuous, because $\chi_s\left( \frac{sT}2 \right) = \varphi(sT) \ \forall s$, and
\[\chi_0(t) = \psi(t)\varphi(t) ,\]
\[\chi_1(t) = \left\{ \begin{array}{lc} \varphi(2t) & t \leq \frac{T}2 \\ \psi\left( 2 \left( t - \frac{T}2 \right) \right) \varphi(T) & t \geq \frac{T}2 \end{array} \right. .\]
\end{proof}

\begin{proof} {\it (Loop):} As $\phi(0) = \phi(1) = \Id$ and $\psi \in \text{SP}(n)$, we can apply the lemma, so $\phi\psi \simeq \phi \diamond \psi$. Thus,
\[\mu_{CZ}(\phi\psi) = \text{deg}(\rho^2 \circ (\widetilde{\psi} \diamond \phi)) = \text{deg}(\rho^2 \circ \widetilde{\psi}) + \text{deg}(\rho^2 \circ \phi) = \mu_{CZ}(\psi) + 2\text{deg}(\rho \circ \phi) .\]
\end{proof}

\begin{proof} {\it (Signature):} Since $S$ is a symmetric matrix, there is an orthogonal matrix $P$ of determinant $1$ such that $PSP^T = \text{diag}(a_1,...,a_{2n})$. Our assumptions imply that $a_i \neq 0$ and $|a_i| < 2\pi \ \forall i$, and therefore the eigenvalues of $J_0 S$ all have norm smaller than $2\pi$. This implies that $\exp(J_0S)$ does not admit $1$ as an eigenvalue, so $\exp(tJ_0S) \in \text{SP}(n)$.

Consider $P_s$ a path of orthogonal matrices with $P_0=P$ and $P_1=\Id$ with $\|P_sSP_s^T\| < 2\pi$. In this case, $\exp(J_0P_sSP_s^T)$ does not admit the eigenvalue $1$ for any $s$, so $\exp(tJ_0PSP^T) \simeq \exp(tJ_0S)$ in $\text{SP}(n)$. Therefore, it is enough to prove the property for the case when $S$ is diagonal.

In this case,
\[J_0S = \begin{pmatrix} 0 & \text{diag}(-a_{n+1},...,-a_{2n}) \\ \text{diag}(a_1,...,a_n) & 0 \end{pmatrix} , \text{ with } |a_i| < 2\pi .\]
We can decompose $(\R^{2n},\Omega_0)$ into a sum of $n$ symplectic planes in a way that $\psi$ decomposes in $\psi_i$ in each of the planes, and such that $\mu_{CZ}(\psi)$ is the sum of all the $\mu_{CZ}(\psi_i)$, for
\[\psi_i(t) = \exp\left(t\begin{pmatrix} 0 & -a_{n+i} \\ a_i & 0 \end{pmatrix}\right) ,\]
by the product property. The closed formula of $\psi_i$ can be computed from the sign of $a_i,a_{n+i}$ and $a_ia_{n+i}$:

\begin{enumerate}
	\item If $a_ia_{n+i} > 0$,
	\[\psi_i(t) = \begin{pmatrix} \cos(\sqrt{a_ia_{n+i}}t) & \mp \sqrt{\frac{a_{n+i}}{a_i}} \sin(\sqrt{a_ia_{n+i}}t) \\ \pm \sqrt{\frac{a_i}{a_{n+i}}} \sin(\sqrt{a_ia_{n+i}}t) & \cos(\sqrt{a_ia_{n+i}}t) \end{pmatrix} \text{ if } a_i \gtrless 0 .\]
	\item If $a_ia_{n+i} < 0$,
	\[\psi_i(t) = \begin{pmatrix} \cosh(\sqrt{-a_ia_{n+i}}t) & \pm \sqrt{-\frac{a_{n+i}}{a_i}} \sinh(\sqrt{-a_ia_{n+i}}t) \\ \pm \sqrt{-\frac{a_i}{a_{n+i}}} \sinh(\sqrt{-a_ia_{n+i}}t) & \cosh(\sqrt{-a_ia_{n+i}}t) \end{pmatrix} \text{ if } a_i \gtrless 0 .\]
\end{enumerate}
In the second case, $\psi_i(t)$ has no eigenvalues on the circle for $t>0$, so $\mu_{CZ}(\psi_i) = 0$ by the zero property.

Otherwise, the eigenvalues of $\psi_i(t)$ are $\cos(\sqrt{a_ia_{n+i}}t) \pm i \sin(\sqrt{a_ia_{n+i}}t)$ so, rotating the plane if needed, we get that
\[ \rho(\psi_i(t)) = \left\{ \begin{array}{lc} e^{i \sqrt{a_ia_{n+i}} t} & \text{if } a_i > 0 \\ e^{- i \sqrt{a_ia_{n+i}} t} & \text{if } a_i < 0 \end{array} \right. .\]
Either way, $\det(\Id - \psi(1)) = 2(1-\cos(a_ia_{n+i})) > 0$, so the extension $\widetilde{\psi_i}$ must end in $W^+ = - \Id$, so $\rho(\widetilde{\psi_i}(2)) = e^{\pm i \pi}$. In the case when $a_i > 0$ $\rho(\widetilde{\psi_i}(t))$ goes from $1=e^0$ to $e^{i\pi}$ counterclockwise, so $\mu_{CZ}(\psi_i) = 1$. On the other hand, when $a_i < 0$ $\rho(\widetilde{\psi_i}(t))$ goes from $1=e^0$ to $e^{-i\pi}$ clockwise, so $\mu_{CZ}(\psi_i) = -1$. Rephrasing, $\mu_{CZ}(\psi_i) = \mathrm{sign}(a_i)$. Therefore,
\[\mu_{CZ}(\psi) = \# \{ i \leq n \ | \ a_i > 0, a_ia_{n+i} > 0 \} - \# \{ i \leq n \ | \ a_i < 0, a_ia_{n+i} > 0 \} = \frac12 \mathrm{Sign}(S) .\]
\end{proof}

\begin{proof} {\it (Determinant):} If $\det(\Id - \psi(1)) > 0$, then $\widetilde{\psi}(2) = W^+$, so $\rho(\widetilde{\psi}(2)) = (-1)^n$. Otherwise, if $\det(\Id - \psi(1)) < 0$, then $\widetilde{\psi}(2) = W^-$ and $\rho(\widetilde{\psi}(2)) = (-1)^{n-1}$. The degree of $\rho^2 \circ \widetilde{\psi}$ is even when $\rho(\widetilde{\psi}) = 1$, and odd when $\rho(\widetilde{\psi}(2)) = -1$. Therefore,
\[(-1)^{n-\mu_{CZ}(\psi)} = \mathrm{sign} ( \det(\Id - \psi(1))) .\]
\end{proof}

\begin{proof} {\it (Inverse):} Recall that $A \in \sympl(\R^{2n},\Omega_0)$ implies that $A^T = \Omega_0 A^{-1} \Omega_0^T = \Omega_0^{-1} A^{-1} \Omega_0$ so, by the naturality property of $\mu_{CZ}$,
\[\mu_{CZ}(\psi^T) = \mu_{CZ}(\psi^{-1}) .\]
Moreover, by the conjugation property of $\rho$ we know that $\rho(A^{-1}) = \overline{\rho(A)} = \rho(A)^{-1}$ for any symplectic matrix. On the other hand, it can be easily checked that $\widetilde{\psi^{-1}} \simeq \widetilde{\psi}^{-1}$. Therefore,
\[\mu_{CZ}(\psi^{-1}) = \text{deg}(\rho^2 \circ \widetilde{\psi^{-1}}) = \text{deg}(\rho^2 \circ \widetilde{\psi}^{-1}) = \text{deg}((\rho^2 \circ \widetilde{\psi})^{-1}) = - \text{deg}(\rho^2 \circ \widetilde{\psi}) = - \mu_{CZ}(\psi) .\]
\end{proof}

Now that we have seen quite in detail the properties of the Conley-Zehnder index, a natural question that arises is if there are other maps satisfying similar properties. However, the answer is that the Conley-Zehnder index is completely determined by (some of) its properties:

\begin{theo} Let $\mu' : \text{SP}(n) \rightarrow \con{Z}$ satisfying the homotopy, loop and signature properties as in \ref{conleyzehnderproperties}. Then, $\mu'(\psi) = \mu_{CZ}(\psi) \ \forall \psi \in \text{SP}(n)$.
\end{theo}

\begin{proof}
Take $\psi \in \text{SP}(n)$. As we observed earlier, there exists $\widetilde{\psi} \in \text{SP}(n)$ with $\widetilde{\psi} \simeq \psi$ and $\widetilde{\psi}(1) = W^{\pm}$. By the homotopy property, $\mu'(\psi) = \mu'(\widetilde{\psi})$.

Let $S^{\pm}$ be such that $W^{\pm} = \exp(\pi J_0S^{\pm})$. In this case, $S^+=\Id$, and
\[S^- = \begin{pmatrix} 0 & 0 & -\frac{\ln 2}{\pi} & 0 \\ 0 & \Id_{n-1} & 0 & 0 \\ -\frac{\ln 2}{\pi} & 0 & 0 & 0 \\ 0 & 0 & 0 & \Id_{n-1} \end{pmatrix} .\]
Consider $\xi(t) = \exp(t\pi J_0S^+)$ (resp with $S^-$ if $\psi(1)=W^-$), and let $\xi^-(t) = \xi(1-t)$. Take $\phi = \widetilde{\psi} \diamond \xi^-$, which is a continuous loop, because $\phi(0) = \phi(1) = W^{\pm}$.

On the other hand, we know that $\widetilde{\psi} \simeq \phi \diamond \xi$. By the lemma \ref{looplemma}, we know also that $\phi \diamond \xi \simeq \phi\xi$. Therefore, $\mu'(\widetilde{\psi}) = \mu'(\phi\xi)$. If we apply now the loop condition, we see that
\[\mu'(\phi\xi) = 2\text{deg}(\rho \circ \phi) + \mu'(\xi) ,\]
and applying the signature condition we see that
\[\mu'(\xi) = \frac12 \mathrm{Sign}(S^{\pm}) .\]
Therefore, we conclude that, for any map $\mu'$ satisfying the homotopy, loop and signature conditions,
\[\mu'(\psi) = 2\text{deg}(\rho \circ \phi) + \frac12 \mathrm{Sign}(S^{\pm}).\]
However, we would get to this very same expression if $\mu'=\mu_{CZ}$. Therefore,
\[\mu'(\psi) = \mu_{CZ}(\psi) .\]
\end{proof}

This is why we can say that the homotopy, loop and signature conditions determine completely the Conley-Zehnder index.
