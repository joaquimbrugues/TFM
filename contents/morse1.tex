\section{Morse functions}

Let us consider a differentiable manifold $M$ without boundary, and a $\mathcal{C}^{\infty}$ - function to $\R$ defined on this manifold. Recall that

\begin{deff}
	A point $p \in M$ is called a {\bf critical point} of the function $f$ if the tangent map $\dd f_p : \straightT_pM \longrightarrow \straightT\R \cong \R$ is zero. In this case, we sat the $f(p)$ is a {\bf critical value}.
\end{deff}

To classify the critical points of a function, we will be interested in the directions in which it is convex and the ones in which it is concave. The best way to study the convexity of a function systematically is its Hessian, or second derivative. However, we find ourselves with a constraint: it is not possible to define the Hessian of a function in all points in a way that it is independent of the choice of the coordinate system! Fortunatelly, it is possible to do it at each critical point, as we will show.

\begin{deff}
	Let $p$ be a critical point of a function $f \in \mathcal{C}^{\infty}(M)$. The {\bf Hessian} of $f$ at $p$ is the bilinear map

\begin{displaymath}
	\begin{array}{rccc} \straightH_p[f] : & \straightT_pM \times \straightT_pM & \longrightarrow & \R \\ & (u,v) & \longmapsto & v(X_u(f)) \end{array} ,
\end{displaymath}

where $X_u$ is a vector field extending $u \in \straightT_pM$ locally.
\end{deff}

To define the extension of a vector to a vector field, we use this lemma

\begin{lema}
	Let $u \in \straightT_pM$. There is a vector field $X_u \in \mathfrak{X}(M)$ such that $X_u(p) = u$.
\end{lema}

This lemma can be proved in a straightforward way using a suitable bump function defined inside a local chart of $p$ in $M$.

\begin{lema}
	The Hessian $\straightH_p[f]$ is well defined (this means, it does not depend on the choice of $X_u$), and it is a bilinear and symmetric map.
\end{lema}

\begin{proof}
	First, we show that it is symmetric, from where it will be clear that it is well defined:

$$\straightH_p[f](v,w) - \straightH_p[f](w,v) = v(X_w(f)) - w(X_v(f)) = \left. X_v \right|_p(X_w(f)) - \left. X_w \right|_p (X_v(f)) = $$ 
$$= \left. [X_v,X_w] \right|_p(f) = \dd_p f \cdot [X_v,X_w](p) = 0 ,$$

where the last term is zero because $\dd_p f = 0$, as $p$ is a critical point for $f$. Thus, if we choose extensions $X_v$ and $X_w$ for $v$ and $w$ (respectivelly), then $\straightH_p[f](v,w) = \straightH_p[f](w,v)$.

Looking again at the intrinsic definition,

$$\straightH_p[f](v,w) = v(X_w(f)),$$

it is obvious that this does not depend on the extension $X_v$ that we choose for $v$, as in the expression only depends on $X_v(p) = v$ regardless of the extension. On the other hand, as we just proved, $\straightH_p[f](v,w) = \straightH_p[f](w,v)$, so, applying the same argument, the Hessian does not depend on the extension chosen for $w$. This proves that the Hessian is well defined.

\

Finally, the Hessian is bilinear, because
$$H_p[f](\alpha u + \beta v, w) = (\alpha u + \beta v)(X_w(f)) = $$
$$= \alpha u(X_w(f)) + \beta v(X_w(f)) = \alpha H_p[f](u,w) + \beta H_p[f](v,w) ,$$

and the same argument applies to the second component by symmetry.
\end{proof}

Notice that the last proof depends on the fact that $p$ is a critical point. This means, in general it is not possible to proof that $\straightH_p[f]$ is well defined, when $p$ is not a critical point.

\begin{rmrk}
	The local form of the Hessian of a function coincides with the Hessian of the local representation of the function in a chart. If $(x_1,...,x_n)$ is a local chart centered in a point $p \in M$ and $\tilde{f}$ is the local representation of $f$ in this chart, then the local expression of $\straightH_p[f]$ is preciselly the matrix

\begin{displaymath}
	\tilde{\straightH}_p[f] := \left( \frac{\partial^2 \tilde{f}}{\partial x_i \partial x_j}(p) \right)_{i,j} .
\end{displaymath}
\end{rmrk}

As we said before, we are interested in the Hessian of a function at a critical point to, in some way, count the number of directions in which the function is convex. To do so, we need the concepts of index and non-degeneracy.

\begin{deff}
	We define the

\begin{itemize}
	\item {\bf Index} of $p$ as the dimension of the maximal subspace $V \subset \straightT_pM$ such that $\left. \straightH_p[f] \right|_V$ is negative definite.
	\item {\bf Nullity} of $p$ as the dimension of the null-space of $\straightH_p[f]$, this means, the maximal subspace $N \subset \straightT_pM$ such that $\straightH_p[f](N,\cdot) = 0$.
	\item {Non-degenerate critical points} of $f$ as the points $p$ that have nullity $0$, this means, that the local representation of $\straightH_p[f]$ has maximal rank in any local chart representation.
\end{itemize}
\end{deff}

Notice that all the definitions that we just gave are well defined, because the index and nullity of a matrix are independent of the basis chosen to represent the matrix, so they are also independent of any change of coordinates.

Therefore, it makes sense to classify the critical points of a manifold according to their index.

\begin{deff}
	We say that a function $f \in \mathcal{C}^{\infty}(M)$ that has only non-degenerate critical points is a {\bf Morse function}.
	If $f$ is a Morse function, we denote

	\begin{displaymath}
		\crit_k(f) = \{p \in M \ | \text{ $p$ has index $k$}\} ,
	\end{displaymath}
	\begin{displaymath}
		\crit(f) = \bigcup_{k \geq 0} \crit_k(f) .
	\end{displaymath}
\end{deff}

Now that we have presented the Morse functions, we are ready to study their behaviour in a neighbourhood of a critical point:

\begin{prop}
	{\bf (Morse Lemma):} Let $p \in \crit(f)$. Then there is a local coordinate system $(U,(y_1,...,y_n))$ centered on $p$ (this means, with $y_i(p) = 0 \ \forall i$) such that, if $k$ is the index of $p$,

$$\left. f \right|_U = f(p) - y_1^2 - ... - y_k^2 + y_{k+1}^2 + ... + y_n^2 .$$
\end{prop}

%Here goes the proof...

\begin{coro}
	There is a unique maximal Morse chart (modulo permutations of the coordinates) for each $p \in \crit_k(f)$.
\end{coro}