\section{Applications to topology} \label{section:morse_topology}

In this section we are going to review some theorems on how to characterize the topology of a manifold using a Morse function or, more specifically, the critical points of a Morse function.

First of all, let us reduce to the simplest case, when a part of a manifold does not contain critical points.

Let $M$ be a smooth manifold, and $f : M \rightarrow \R$ a smooth function. For $x \in \R$, let $M_x := \{p \in M \ | \ f(p) \leq x\}$.

\begin{prop} \label{nocriticalpoints}
Consider $a, b \in \R$ such that $f^{-1}([a,b]) \subset M$ is compact and does not contain any critical point. Then, $M_a$ is a deformation retract of $M_b$ and, moreover, $M_a \cong M_b$, this means, they are diffeomorphic.
\end{prop}

\begin{proof}
Let $W \subset M$ the open set of non-critical points of $f$, and consider $g$ a Riemannian metric on $M$. We are going to use $g$ to construct a vector field that yields an appropriate flow, which will serve to construct our desired diffeomorphism.

Take $X = \frac1{\|\grad f\|^2}\grad f \in \mathfrak{X}(W)$, and let $\gamma : I \rightarrow M$ be a maximal integral curve of $X$. As a consequence of the definition, we have that
\[\frac{d}{d t} f(\gamma(t)) = d f(\gamma(t)) \cdot \gamma'(t) = d f(\gamma(t)) \cdot X(\gamma(t)) =\]
\[g\left(\grad f(\gamma(t)),X(\gamma(t))\right) = \frac1{\|\grad f(\gamma(t))\|^2} g(\grad f(\gamma(t)),\grad f(\gamma(t))) = 1 .\]
Therefore, if we assume $0 \in I$, we have that $f(\gamma(t)) = f(\gamma(0)) + t$.

Let $K = f^{-1}([a,b]) \subset W$, which is compact by hypothesis. Take the initial condition $\gamma(0) \in f^{-1}(a)$. There are two cases in which we can find ourselves: either $\gamma(t) \in K \ \forall t \in I, t > 0$, or the solution goes out of $K$ after some time:
\begin{itemize}
	\item If $\gamma(t) \in K \ \forall t > 0$, then the solution is defined inside a compact set. Therefore, it is defined for all positive time, so $[0,+\infty) \subset I$. In particular, the solution is defined in $[0,b-a]$.
	\item If there is $s \in I, s > 0$ such that $\gamma(s) \notin K$, then $b < f(\gamma(s)) = f(\gamma(0)) + s = a+s$, so $s > b-a$. Therefore, $[0,b-a] \subset I$.
\end{itemize}

Moreover, we can extend $X$ to the whole manifold without losing the properties that we just announced. Take a bump function $\psi : M \rightarrow \R$, such that
\begin{enumerate}
	\item $\left. \psi \right|_K = 1$.
	\item Its support is contained in $W$.
\end{enumerate}

Then, we can construct the vector field $Y$ on the whole manifold by
\[Y = \left\{ \begin{array}{ll} \psi(x) X(x) & \text{if } x \in W \\ 0 & \text{otherwise} \end{array} \right. ,\]
and it coincides with $X$ at $K$, so all the results that we proved remain true. Let $\varphi^t$ be the flow of $Y$. If necessary, we can shrink the support of $\psi$ to guarantee that the flow is defined up to time $b-a$, so $\varphi^{b-a}$ is a well defined diffeomorphism on $M$ that sends $M_a$ onto $M_b$. This concludes the proof that $M_a \cong M_b$.

To prove that $M_a$ is a deformation retract of $M_b$, consider the collection of maps
\[\begin{array}{rccc} r : & M_b \times [0,1] & \longrightarrow & M_b \end{array}\]
defined by

\[r(x,t) = \left\{ \begin{array}{ll} x & \text{if } f(x) \leq a \\ \varphi^{t(a-f(x))}(x) & \text{if } a \leq f(x) \leq b \end{array} \right. ,\]
which induce the desired retraction.
\end{proof}

There is an immediate corollary to this proposition, that is also a classic theorem on differential geometry.

\begin{coro} \label{coro:reeb}
{\bf (Reeb's theorem):} Let $M$ be a compact smooth manifold. Suppose that there is a Morse function $f : M \rightarrow \R$ that has only two critical points. Then, $M$ is homeomorphic to a sphere.\end{coro}

\begin{rmrk} A smooth function over a compact manifold must always have at least two critical points (the maximum and the minimum) because of Weierstrass theorem. The theorem, in a way, strengthens this assertion: the {\it only} (up to homeomorphism) manifolds that admit functions with only two critical points are the spheres.
\end{rmrk}

\begin{proof}
Normalize the function so that $f(M) = [0,1]$. By the Morse lemma (\ref{morselemma}), there is $\e > 0$ small enough so that $f^{-1}([0,\e]) = M_{\e}$ and $f^{-1}([1-\e,1])$ are diffeomorphic to disks in $\R^n$ (with $n$ the dimension of the manifold).

By the proposition (\ref{nocriticalpoints}) that we just proved, we know that $M_{\e}$ is diffeomorphic to $M_{1-\e}$, so it is also an open disk. Therefore, $M$ is diffeomorphic to two disks glued together by their boundaries.

Let $\varphi$ denote the gluing map between the boundaries of the disks, so we may write $M \cong \con{D}^n \cup_{\varphi} \con{D}^n$. Then, if we denote the standard sphere by $\con{S}^n = \con{D}^n \cup_{\text{Id}} \con{D}^n$ (this means, the standard sphere is the result of gluing two disks together with a trivial gluing map), we can construct an explicit homeomorphism
\[\begin{array}{rccc} h : & \con{S}^n \cong \con{D}_1^n \cup_{\text{Id}} \con{D}_2^n & \longrightarrow & \con{D}_1^n \cup_{\varphi} \con{D}_2^n \end{array}\]
by
\[h(z) = \left\{ \begin{array}{ll} z & \text{ if } z \in \con{D}_1^n \\ \|z\| \varphi\left( \frac{z}{\|z\|} \right) & \text{ if } z \in \con{D}_2^n \backslash \{0\} \\ 0 & z = 0 \in \con{D}_2^n \end{array} \right. .\]
Therefore, $M \cong \con{S}^n$.
\end{proof}

Now we can move to a more interesting result, that describes the topology of a manifold when crossing a critical point:

\begin{theo} \label{theo:morse_topology}
Let $p \in M$ be a non-degenerate critical point of a smooth function $f : M \rightarrow \R$. Let $k$ be its index, and let $c = f(p)$. Take $\e > 0$ small enough such that $f^{-1}([c-\e,c+\e])$ is compact and does not contain any critical point different of $p$.

Then, $M_{c+\e} \simeq M_{c-\e} \cup \con{D}^k$, this means, $M_{c+\e}$ is homotopically equivalent to $M_{c-\e}$ with a $k$-cell adjointed.
\end{theo}

\begin{proof}
First of all we will construct a function $F$ with $F < f$ in a neighbourhood of $p$ and such that it coincides with $f$ elsewhere.

Let $(U,u_1,...,u_n)$ be a Morse chart centered on $p$, so the function $f$ is
\[\left. f \right|_U = c - u_1^2 - \cdots - u_k^2 + u_{k+1}^2 + \cdots + u_n^2 .\]

We can take a chart and $\e$ such that the image of $U$ under the chart contains the closed ball $B = \{(u_1,...,u_n) \in \R^n \ | \ u_1^2 + \cdots + u_n^2 \leq 2\e\}$. On the other hand, consider the $k$-disk inside $B$,
\[e^k = \{ (u_1,...,u_n) \in \R^n \ | \ u_1^2 + \cdots + u_k^2 \leq \e \text{ and } u_{k+1} = \cdots = u_n = 0 \} .\]

We will use $e^k$ to denote both the disk as we have defined it and its preimage by the chart $(u_1,...,u_n)$. Thus, $e^k$ is precisely a cell attached to $M_{c-\e}$, because $M_{c-\e} \cap e^k = \partial e^k$. What we need to prove, then, is that $M_{c-\e} \cup e^k$ is a deformation retract of $M_{c+\e}$.

Let
\[\xi = u_1^2 + \cdots + u_k^2\]
\[\eta = u_{k+1}^2 + \cdots + u_n^2\]
so that
\[\left. f \right|_U = c - \xi + \eta .\]
We will use this decomposition to define a function F as we said before. To this end, take $\mu : [0,+\infty) \rightarrow \R$ a smooth function such that

\begin{enumerate}
	\item $\mu(0) > \e$.
	\item $\mu(t) = 0 \ \forall t \geq 2\e$.
	\item $-1 < \frac{d \mu}{d t}(t) \leq 0 \ \forall t \in (0,+\infty)$.
\end{enumerate}

Then, let $F$ be defined by
\[F(q) = \left\{ \begin{array}{ll} f(q) & \text{if } q \notin U \\ f(q) - \mu(\xi(q) + 2\eta(q)) & \text{if } q \in U \end{array} \right. ,\]
so
\[\left. F \right|_U = c - \xi + \eta - \mu(\xi + 2\eta) .\]
$F$ is well defined, because $U$ contains the closed ball $\{ \xi + \eta \leq 2\e \}$.

Let us break the rest of the proof in 4 steps.

\

{\bf Step 1:} See that $F^{-1}((-\infty,c+\e]) = M_{c+\e} \ \left( = f^{-1}((-\infty,c+\e]) \right)$.

Notice, first of all, that $F$ and $f$ coincide outside of the region $E:=\{ \xi + 2\eta \leq 2\e\}$, so it suffices to show that $F^{-1}((-\infty,c+\e]) \cap E = f^{-1}((-\infty,c+\e]) \cap E$. But notice that, if $q \in E$,
\[F(q) \leq f(q) = c - \xi + \eta \leq c + \frac12 \xi + \eta = c + \frac12 (\xi + 2\eta) \leq c + \e ,\]
so, in fact, $E \subset F^{-1}((-\infty,c+\e])$, and $F^{-1}((-\infty,c+\e]) \subset f^{-1}((-\infty,c+\e])$. Therefore, $F^{-1}((-\infty,c+\e]) = f^{-1}((-\infty,c+\e])$.

\

{\bf Step 2:} See that $F$ and $f$ have exactly the same critical points.

We can express $F$ as a function of $\xi$ and $\eta$, so we can say that
\[d F = \frac{\partial F}{\partial \xi} d \xi + \frac{\partial F}{\partial \eta} d \eta .\]
Where

\begin{enumerate}
	\item $\frac{\partial F}{\partial \xi} = -1 - \mu'(\xi+2\eta) < 0$ by the definition of $\mu$.
	\item $\frac{\partial F}{\partial \eta} = 1 - 2 \mu'(\xi+2\eta) \geq 0$ by the definition of $\mu$.
	\item $d \xi$ and $d \eta$ are simultaneously zero if and only if $\xi = \eta = 0$, this means, at $p$.
\end{enumerate}

Therefore, the only critical point of $F$ in $U$ is $p$, so $F$ and $f$ have exactly the same critical points.

\

{\bf Step 3:} See that $F^{-1}((-\infty,c-\e])$ is a deformation retract of $f^{-1}((-\infty,c+\e])$.

The fact that $F \leq f$, together with the claim proved in Step 1, allows us to deduce that $F^{-1}([c-\e,c+\e]) \subset f^{-1}([c-\e,c+\e])$. In particular, this means that $F^{-1}([c-\e,c+\e])$ is compact. Moreover, notice that the critical value of $F$ at $p$ satisfies that
\[F(p) = c - \mu(0) < c - \e ,\]
because of the definition of $\mu$. Therefore, $p \notin F^{-1}([c-\e,c+\e])$, so this set does not contain critical points. With all these considerations, we see that we are under the hypothesis of proposition \ref{nocriticalpoints}. This means that $F^{-1}((-\infty,c-\e])$ is a deformation retract of $F^{-1}((-\infty,c+\e]) = f^{-1}((-\infty,c+\e]) = M_{c+\e}$ (and, in fact, they are diffeomorphic).

Let $H = \overline{F^{-1}((-\infty,c-\e]) \backslash M_{c-\e}}$. With this notation, $F^{-1}((-\infty,c-\e]) = M_{c-\e} \cup H$.

Actually, we have seen that $M_{c+\e} \simeq M_{c-\e} \cup H$.

\

{\bf Step 4:} See that $M_{c-\e} \cup e^k$ is a deformation retract of $M_{c-\e} \cup H$.

In the variables $(\xi, \eta)$ the disk $e^k$ in $M$ can be expressed as $e^k = \{q \in U \ | \ \xi(q) \leq \e \text{ and } \eta(q) = 0\}$. We claim that $e^k \subset H$.

First of all, $e^k \subset F^{-1}((-\infty,c-\e])$. This can be seen because, if $q \in e^k$,
\[F(q) = c -\xi(q) - \mu(\xi(q)) \leq c - \mu(0) \leq c - \e .\]
In the first inequality we have used the fact that $\xi \geq 0$ and that $\mu$ is a decreasing function. In the second inequality we used that $\mu(0) \geq \e$.

On the other hand, $f(q) = c - \xi \geq c - \e$, with an equality only when $\xi = \e$, this means, at $\partial e^k$. Therefore, as we claimed, $e^k \subset H$.

Now we can construct a retraction of $M_{c-\e} \cup H$ onto $M_{c-\e} \cup e^k$. Let us call it $r_t$. Let $r_t$ be the identity ouside of $U$ for all $t$, and separate $U \cap \left( M_{c-\e} \cup H \right)$ in three regions:
\[C_1 = \{ q \ | \ \xi(q) \leq \e \} ,\]
\[C_2 = \{ q \ | \ \e \leq \xi(q) \leq \eta(q) + \e \} ,\]
\[C_3 = \{ q \ | \ \eta(q) + \e \leq \xi(q) \Leftrightarrow f(q) \leq c - \e \} .\]
We will construct $r_t$ separately on each of these three regions, and we will prove that it is the desired retraction.

\begin{itemize}
	\item $r_t$ on $C_1$. We define
	\[r_t(u_1,...,u_k,u_{k+1},...,u_n) = (u_1,...,u_k,tu_{k+1},...,tu_n) ,\]
	or, equivalently, $r_t(\xi,\eta) = (\xi,t^2\eta)$. It is clear that $r_1$ is the identity, and that $r_0$ is a projection onto $e^k$. Moreover, $F(r_t(q)) \leq c - \e$, because $\frac{\partial F}{\partial \eta} > 0$.
	\item $r_t$ on $C_2$. We define
	\[r_t(u_1,...,u_n) = (u_1,...,u_k,s_tu_{k+1},...,s_tu_n) ,\]
	or, as before, $r_t(\xi,\eta) = (\xi, s_t^2 \eta)$. We define
	\[s_t = t + (1-t) \sqrt{\frac{\xi-\e}{\eta}} .\]
	It is clear that $r_1$ is the identity. On the other hand, notice that
	\[f(r_0(q)) = f(\xi,s_0^2\eta) c - \xi + s_0^2 \eta = c - \xi + \xi - \e = c -\e,\]
	so $r_0$ maps all of $C_2$ onto the boundary of $M_{c-\e}$.
	\item On $C_3$, we let $r_t = \text{Id}$ for all $t$. When $\xi - \e = \eta$, it coincides with the last definition.
\end{itemize}

We need to check that $r_t$ is continuous. In particular, we need to check it when $\xi \rightarrow \e$ and $\eta \rightarrow 0$. First of all, notice that

\begin{itemize}
	\item when $\xi = \e$, $s_t = t$,
	\item when $\xi - \e = \eta$, $s_t = 1$.
\end{itemize}

Thus, the only points where it is not clear if $r_t$ is continuous are those such that $\xi = \e$ and $\eta = 0$. In particular, we are to check the continuity in the region $C_2$. In this case, however, we have that
\[\xi - \e \leq \eta \Rightarrow 0 \leq \frac{\xi - \e}{\eta} \leq 1,\]
so $s_t$ stays bounded in the whole $C_2$. Moreover, for each $i > k$, the coordinate $u_i$ is mapped as $u_i \mapsto s_t u_i$. In addition, $|u_i| \leq \eta$. Taking all of this into account, we deduce that
\[0 \leq |s_tu_i| \leq s_t\eta \xrightarrow[\eta \rightarrow 0, \xi \rightarrow \e]{} 0 ,\]
so, in particular, $s_tu_i \xrightarrow[\eta \rightarrow 0, \xi \rightarrow \e]{} 0$, as we wanted to see.

Thus, $r_t$ is continuous, so it is a retraction from $M_{c-\e} \cup H$ onto $M_{c-\e} \cup e^k$. This concludes the proof.
\end{proof}
