\section{The Arnold conjecture}

In this section we focus on the dynamical aspect of symplectic manifolds. In particular, we are going to be interested in fixed points and periodic orbits of the flow for a given Hamiltonian function. Let us begin with an elementary result.

\begin{prop}
A point $p \in M$ is a fixed point for the flow $\varphi_{X_H}^t$ for all time if and only if it is a critical point of $H$.
\end{prop}

\begin{proof}
Let $p$ be a critical point of $H$, so $d H(p) = 0$. As $\omega$ is nondegenerate at each point, we get that $X_H(p) = 0$. Conversely, if $X_H(p) = 0$ then necesarily $d H(p) = 0$ and therefore $p$ is a critical point of $H$.
\end{proof}

Using Morse theory, it is then quite clear that, if $H$ is a Morse function, we have that
\[\# \{\text{fixed points of } X_H\} \geq \sum_i \text{dim}H_i(M) ,\]
because the fixed points of $X_H$ coincide with the critical points of $H$.

Of course, the number of fixed points of a system may be infinite (but only if the manifold is not compact: recall that critical points of a Morse function are isolated by proposition \ref{morselemma}), but if they are finite there is a constraint on the minimum number of them, and this constraint does not depend on the properties of $H$ but on the topology of the manifold.

The idea of Arnold's conjecture is to translate this result to the 1-periodic orbits of $\varphi_{X_H}^t$, this means, the solutions $x : [0,1] \rightarrow M$ of $\dot{x} = X_H$ such that $x(0) = x(1)$, or, rephrasing,
\[\varphi_{X_H}^1(x(0)) = x(0) .\]
It is clear that a fixed point is an orbit, so we can propose the following (already proved) theorem:

\begin{theo}
The number of periodic orbits of a Hamiltonian system's solution is greater or equal than the sum of dimensions of the homology groups of $M$.
\end{theo}

However, we can ask ourselves a far more general question: does this result still hold when the system is not autonomous? This mean, if we take a time dependent Hamiltonian $H_t(x)$, is there a lower bound on the number of periodic solutions to the system
\[\dot{x}(t) = X_{H(t)}(x(t)) .\]

Of course, a generalization of the result for fixed points does not make sense anymore: for each symplectic manifold $M$ it is possible to produce some time-dependent Hamiltonian $H_t$ with no fixed points. Nonetheless, the question about the 1-periodic orbits is not so clear. To tackle it, we need to restrict slightly our object of interest to the periodic solutions that are nondegenerate:

\begin{deff}
Let $\varphi^t$ the flow of a time-dependent Hamiltonian. Consider $x$ a periodic solution of $X_{H(t)}$. We say that $x$ is a {\bf nondegenerate periodic orbit} if
\[\det(\straightT_{x(0)} \varphi^1 - \text{Id}) \neq 0 ,\]
this means, if $\straightT_{x(0)} \varphi^1$ does not have $1$ as an eigenvalue.
\end{deff}

Under this condition we do have the following result:

\begin{theo}
{\bf (Arnold's conjecture):} Let $(M,\omega)$ be a compact symplectic manifold, and let $H_t$ be a time-dependent Hamiltonian. Then, the number of nondegenerate periodic orbits of its flow is greater or equal than
\[\sum_i \text{dim}H_i(M) .\]
\end{theo}

Despite the fact that we call it a conjecture, it has been already proved, although the proof for any compact symplectic manifold $M$ took some time to be found. Floer did a remarkable breakthrough proposing his homology, which we are going to study, to find a proof for aspherical manifolds (a notion that we are going to present in the next section).

As before, the most surprising thing about this result is the fact that the lower bound does not depend on $H_t$, but only on the topology in which the system is defined.
