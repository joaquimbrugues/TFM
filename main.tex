\documentclass[a4paper,11pt]{book}
\usepackage[utf8]{inputenc}
\usepackage[english]{babel}
\usepackage{import}
\usepackage{styles/TFG}
\usepackage{styles/TFM}
\usepackage[hidelinks,pdfauthor={Joaquim Brugues Mora},pdftitle={Master thesis Floer Homology},pdftex]{hyperref}

\begin{document}

%----------------- Title page --------------------%
\begin{titlepage}
	\centering
	{\scshape\LARGE Universitat Politècnica de Catalunya - BarcelonaTech\par}
	{\scshape\LARGE Facultat de Matemàtiques i Estadística, \par}
	\vspace{1cm}
	{\scshape\Large Master Thesis\par}
	\vspace{1.5cm}
	{\huge\bfseries Floer Homology\par}
	\vspace{2cm}
	{\Large\itshape Joaquim Brugués Mora\par}
	\vfill
	advisor\par
	{\Large\itshape Eva Miranda Galceran \\ Cédric Oms \par}

	\vfill

% Bottom of the page
	{\large \today\par}
\end{titlepage}

%----------------- Summary --------------------%

\chapter*{Summary}

%TODO: Here we should put the summary and keywords for this project.

%----------------- Table of contents --------------------%
\tableofcontents

\mainmatter

%Chapter 1, on Morse theory
\chapter{Fundamentals on Morse homology}
\fancyhead[LE]{Morse Theory}
In this chapter we explain the basic constructions on Morse theory. Our aim is not to get a deep understanding of the theory, but to get familiar with the way how the results are proved and which kind of tools are used. We do so because Floer homology will use concepts that are analogous to the ones presented here (but, of course, in a more complex setting), so it is convenient to be familiar with Morse homology before jumping to the Floer homology. A detailed presentation of Morse homology can be found in \cite{milnor1963morse}, and in the first chapters of \cite{audin2014morse}.

\fancyhead[RO]{Morse functions}
\import{contents/}{morse1.tex}

\fancyhead[RO]{Applications to topology}
 \import{contents/}{morse4.tex}

\fancyhead[RO]{The Morse complex}
 \import{contents/}{morse2.tex}

\fancyhead[RO]{Pseudogradients and the Smale condition}
 \import{contents/}{morse3.tex}

\fancyhead[RO]{The differential on the Morse complex}
 \import{contents/}{morse5.tex}

\fancyhead[RO]{The Morse homology is well defined}
 \import{contents/}{morse6.tex}

%Chapter 2, symplectic geometry
 \chapter{Symplectic manifolds and the Arnold conjecture}
\fancyhead[LE]{Symplectic manifolds and the Arnold conjecture}
In this chapter we are going to present the context in which Floer homology arises, which is the one of symplectic manifolds. Roughly speaking, the structure of a symplectic manifold arises when trying to generalise the dinamics of Hamiltonian systems from $\R^{2n}$ to a $2n$-dimensional manifold. A thorough description of the theory can be found in \cite{da2001lectures}, and we will assume that the reader is familiar with the basic concepts.

We will also present the first motivation for the development of Floer theory, which is the Arnold conjecture on fixed points of symplectomorphisms.

\fancyhead[RO]{A toolkit of symplectic geometry}
\import{contents/}{symplectic1.tex}

\fancyhead[RO]{The Arnold conjecture}
\import{contents/}{symplectic2.tex}

\fancyhead[RO]{The conditions on the manifold}
\import{contents/}{symplectic3.tex}

 %TODO: Add figures to Chapter 1, and revise everything. Complete the proof of the appendix (that the map is injective).

 %TODO: Reference with the proof that Morse homology is equivalent to cellular homology, and mention something about algebraic topology

 % Chapter 3, the Floer equation
 \chapter{The Floer equation}
 \fancyhead[LE]{The Floer equation}
 In this chapter we are going to define the first constructions needed to define the Floer complex, namely the loop space, the action functional and the flow of the Floer equation. In every step we are going to refer to the Morse homology and the steps proposed at the section \ref{section:morse_complex}.

 \fancyhead[RO]{The space of contractible loops}
 \import{contents/}{floereq1.tex}

 \fancyhead[RO]{The action functional}
 \import{contents/}{floereq2.tex}

 %TODO: Chapter with the action functional and deduction of the Floer equation

 %TODO: Chapter about the Maslov/Conley-Zehnder index

 %TODO: Appendix about Banach manifolds, the topology on LM, and the Sobolev spaces

%----------------- Appendices ---------------------%
\begin{appendices}
\fancyhead[LE,RO]{Appendices}
\chapter{Additional results on Morse theory}
\import{contents/}{morse7.tex}
\end{appendices}

%----------------- Bibliography --------------------%
%Here goes the bibliography of the project
\bibliographystyle{alpha}
\bibliography{references}
\addcontentsline{toc}{chapter}{Bibliography}
\end{document}
