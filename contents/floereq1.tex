\section{The space of contractible loops} \label{section:floerq_loops}

In Section \ref{section:morse_complex} we outlined the crucial steps to construct the complex to study the critical points of a given function $f$ in a manifold $M$. Here, we are to tackle the first step in the list, this means, define our object of study and the space it is contained into.

Let $(M,\omega)$ be a compact symplectic manifold, and let $H_t$ a $1$-periodic function (in $t$) defined on $\R \times M$. The object that we are interested in are the 1-periodic solutions of
\[\dot{x}(t) = X_{H_t}(x(t)) ,\]
this means, the smooth maps $x : \con{S}^1 \rightarrow M$ satisfying the previous equation. Thus, it seems that the solutions must be looked for in the space
\[\mathcal{C}^{\infty}(\con{S}^1,M) = \{ x : \con{S}^1 \rightarrow M \ | \ x \text{ smooth} \} .\]
However, this space may not be connected: any two non-homotopic loops belong to different connected components. This means that we need to restrict ourselves to a connected component. As we admit constant solutions (this means, fixed points of $X_{H_t}$), it is natural that we consider the connected component that contains the trivial loops, this means, the space of contractible loops:

\begin{deff}
The {\bf space of contractible loops} in $M$ is
\[\mathcal{L}M = \{ x \in \mathcal{C}^{\infty}(\con{S}^1,M) \ | \ x \text{ contractible} \} ,\]
with the $\mathcal{C}^{\infty}$-topology\footnote{See the 2nd chapter of \cite{hirsch2012differential} for more details on the topology of this space.}.
\end{deff}

We will not talk much about the structure of $\mathcal{L}M$ as a manifold, as we mostly will use it in a formal manner. It is important, however, to identify the tangent space $T_x\mathcal{L}M$ at a loop $x$. This will be a normed space (not necessarily complete, though) of the tangency classes of curves from $(-\e,\e)$ to $\mathcal{L}M$. This means, we are interested in the derivative at $0$ (with respect to $s$) of a map
\[u : \con{S}^1 \times (-\e,\e) \longrightarrow M .\]
It is clear that this is a section of the vector bundle $x^{\ast}TM$. This way, we can define

\begin{deff}
The {\bf tangent space} of $\mathcal{L}M$ at a loop $x$ is
\[T_x\mathcal{L}M = \Gamma(x^{\ast}TM) = \{ Y : \con{S}^1 \rightarrow TM \ | \ Y \text{ smooth, } \pi \circ Y = x\},\]
where $\pi : TM \rightarrow M$ is the canonical projection of the tangent bundle.
\end{deff}

(This is not strictly a definition, but a characterization of the tangent bundle $T\mathcal{L}M$).
