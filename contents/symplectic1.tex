\section{Introduction to symplectic manifolds}

Let $(\vc{q},\vc{p})$ denote the coordinates of the real space $\R^{2n}$. Consider $H : \R^{2n} \rightarrow \R$ a smooth function. The Hamiltonian system induced by the function $H$ is the system of differential equations

$$\frac{\dd p_i}{\dd t} = - \frac{\partial H}{\partial q_i}$$

$$\frac{\dd q_i}{\dd t} = \frac{\partial H}{\partial p_i} \end{array} .$$

This systems of equations arise naturally in many problems of applied mathematics. An start point of symplectic geometry consists in attempting to generalize this kind of construction to smooth manifolds, and to do so in well defined way (this means, independent of the choice of coordinates).

Let $M$ be a smooth manifold, which we will assume to be $2n$-dimensional for reasons that will be clear soon.
