This chapter will give the guidelines of how the Floer complex is constructed. In particular, we will define the differential on the chain complex, and comment on the invariance of the resulting homology.

\section{Linearization of the Floer equation}

We will use the Conley-Zehnder index throughout this chapter. There is an introduction to what is this map and how to define it in the Appendix \ref{chapter:conley_zehnder}. For the purposes of this chapter, it is enough to comment that it is a map that associates an integer number to each path of symplectic matrices starting from $\Id$ and ending to a non-degenerate symplectic matrix (by this we mean a symplectic that does not have the eigenvalue $1$). In an abuse of notation, for $x \in \mathcal{L}M$ we will denote by $\mu_{CZ}(x)$ the Conley-Zehnder index of the differential of the flow along $x$, $d\varphi_{X_H}^t(x(t))$.

\begin{deff} \label{deff:floer_complex}
Let $H_t$ be a non-degenerate Hamiltonian. The {\bf $k$-th group of the Floer complex} is the $\con{Z}_2$-module generated by the periodic solutions of $X_H$ with Conley-Zehnder index $k$:
\[CF_k(M,H,J) = \langle \left\{ \gamma : \con{S}^1 \rightarrow M \ | \ \dot{\gamma} = X_H(\gamma) \text{ and } \mu_{CZ}(\gamma) = k \right\} \rangle_{\con{Z}_2} .\]
\end{deff}

In order the define a complex chain we need to provide a differential. In the case of the Morse complex we used pseudogradients in order to connect critical points and, from that, count solutions of the flow and use them to construct the differential. In the case of the Floer complex, we will use the solutions to the Floer equation \ref{equation:floer_equation}. We already know (by the theorem \ref{theo:floer_endpoints}) that the solutions to this equation connect the critical points of the action functional, this means, the periodic orbits of $X_H$. Moreover, we have seen (in the theorem \ref{floer_compact}) that the set of solutions is compact. Therefore, we just need to find the way to count these solutions in the case of consecutive indices. In order to do this, we must look at the linearized version of the Floer equation. We need to begin by understanding in which space we need to define it.

\begin{deff} \label{deff:floer_operator}
The {\bf Floer operator} is the function
\[\begin{array}{rccc} \mathcal{F} : & \mathcal{C}^{\infty}_{\mathrm{loc}}(\R\times\con{S}^1,M) & \longrightarrow & \mathcal{C}^{\infty}_{\mathrm{loc}}(\R\times\con{S}^1,M) \\ & u & \longmapsto & \frac{\partial u}{\partial s} + J(u) \frac{\partial u}{\partial t} + \grad H (u) \end{array} .\]
\end{deff}

In order to linearize this operator, we have to look at a space of perturbations of the solutions $u : \R \times \con{S}^1 \rightarrow M$. In particular, we will look for the Sobolev-kind of functions that result from perturbing $u$. The resulting space will be a Banach manifold, this means, a topological space that is locally diffeomorphic to a Banach space. A good introduction to the theory of Banach manifolds can be found at \cite{abraham2012manifolds}, although the general idea is sufficient for the purposes of this chapter.

\begin{deff}
Let $\mathcal{M}(x,y)$ denote the space of solutions of the Floer equation such that
\[\lim_{s \rightarrow -\infty} u(s,t) = x, \lim_{s \rightarrow +\infty} u(s,t) = y.\]
Then, we define the Banach manifold
\[\mathcal{P}^{1,p}(x,y) = \left\{P : (s,t) \mapsto \exp_{w(s,t)}Y(s,t) \ | \ w \in \mathcal{M}(x,y) \text{ and } Y \in W^{1,p}(w^{\ast}(TM))\right\} ,\]
where $W^{1,p}(w^{\ast}(TM))$ denotes the space of fibers of $w^{\ast}(TM)$ that belong to the Sobolev space $W^{1,p}$, for $p > 2$. By this, we mean that $Y$ is a map
\[\begin{array}{rccc} Y : & \R \times \con{S}^1 & \rightarrow & TM \end{array}\]
with $\pi \circ Y = w$.
\end{deff}

\begin{rmrk}
The fact that $p > 2$ implies that $Y$ are continuous maps, so the elements of $\mathcal{P}^{1,p}(x,y)$ are continuous.
\end{rmrk}

In this Banach manifold, the Floer operator has a natural extension
\[\begin{array}{rccc} \mathcal{F} : & \mathcal{P}^{1,p}(x,y) & \rightarrow & L^p(\R \times \con{S}^1, TM) \\ & u & \longmapsto & \frac{\partial u}{\partial s} + J(u)\frac{\partial u}{\partial t} + \grad H(u) \end{array} ,\]

and we can compute its linearization (assuming that we are inside some local chart of $\mathcal{P}^{1,p}(x,y)$:

\[\mathcal{F}(u+Y) = \frac{\partial (u + Y)}{\partial s} + J(u) \frac{\partial (u + Y)}{\partial t}  + \grad H(u + Y) ,\]

which leads to
\[d\mathcal{F}_u(Y) = \frac{\partial Y}{\partial s} + J(u) \frac{\partial Y}{\partial t} + dJ_u(Y) \frac{\partial u}{\partial t} + d(\grad H(u)) Y\]

\begin{prop}
Let us denote by $S(s,t)$ the linear operator such that $S(s,t)Y = dJ_u(Y) \frac{\partial u}{\partial t} + d(\grad H(u))Y$. Then, there exist the limits
\[\lim_{s \rightarrow \pm \infty} S(s,t) = S^{\pm}(t) ,\]
where $S^{\pm}(t)$ are symmetric matrices, and
\[\lim_{s \rightarrow \pm \infty} \frac{\partial S}{\partial s} (s,t) = 0 .\]

Moreover, the equations $\frac{\partial Y}{\partial t} = J S^{\pm}(t)Y$ are the linearizations of $\dot{x} = X_H(x)$ and $\dot{y} = X_H(y)$, respectively.
\end{prop}

Therefore, the matrices $S^{\pm}$ carry all the relevant information in order to compute the Conley-Zehnder indexes of $x$ and $y$. In particular, we will be able to use the values of the indices to compute the dimension of $\mathcal{M}(x,y)$.

\section{Fredholm operators. Regularity}

\begin{deff}
A linear map $L : E \rightarrow F$ between Banach spaces is a {\bf Fredholm operator} if its kernel is finite dimensional and its image has finite codimension. In this case, its index is
\[\indx(L) = \dim \mathrm{Ker} L - \dim \mathrm{CoKer} L .\]
If $\mathcal{F} : E \rightarrow F$ is a smooth map between Banach manifolds, we say that it is Fredholm if $d\mathcal{F}_x$ is Fredholm for every $x \in E$, and define its index as the index of its linearization.
\end{deff}

Fredholm operators are essential in the theory of partial differential equations. In our context, we want to make use of an specific result regarding Fredholm maps:

\begin{theo} \label{theo:local_surjection} (Local surjection theorem): Let $\mathcal{F} : E \rightarrow F$ a Fredholm map between Banach manifolds and let $y \in F$ such that $d\mathcal{F}_x$ is surjective $\forall x \in \mathcal{F}^{-1}(y)$. Then, $\mathcal{F}^{-1}(y)$ is a smooth manifold of dimension $\indx(\mathcal{F})$, and its tangent space at $x$ is $\mathrm{Ker}(d\mathcal{F}_x)$.
\end{theo}

This theorem gives us the link that we wanted: if the Floer operator is a Fredholm operator, we are able to compute its index, and we are able to show that $d\mathcal{F}_u$ is surjective, then we know the dimension of $\mathcal{M}(x,y)$.

To meet the first requirement, we have that

\begin{theo} \label{theo:floer_fredholm}
For all $u \in \mathcal{M}(x,y)$ the operator $d\mathcal{F}_u$ is Fredholm, and
\[\indx(d\mathcal{F}_u) = \mu_{CZ}(x) - \mu_{CZ}(y) .\]
\end{theo}

For the second requirement, we have to make use of the notion of regularity of a pair $(H,J)$.

\begin{deff}
A pair $(H,J)$ of a time-dependent Hamiltonian and an almost complex structure over $M$ is said to be {\bf regular} if, for all $u \in \mathcal{M}$, the linearized Floer operator
\[\begin{array}{rccc} d\mathcal{F}_u : & \mathcal{P}^{1,p}(x,y) & \rightarrow & L^p(\R\times\con{S}^1,M) \end{array}\]
is surjective.
\end{deff}

\begin{theo} \label{theo:regular}
For any non-degenerate Hamiltonian $H_0$ and for any almost complex structure $J$, there is a dense Banach space $\mathcal{H} \subset \mathcal{C}^{\infty}(\con{S}^1 \times M, \R)$ (with the $\mathcal{C}^1$ topology) and subset $\mathcal{H}_{\mathrm{reg}}$ such that:
\begin{enumerate}
	\item $\mathcal{H}_{\mathrm{reg}}$ is the intersection of a countable family of open dense subsets of $\mathcal{H}$.
	\item $\mathcal{H}_{\mathrm{reg}}$ contains an open neighbourhood of $0$.
	\item For all $h \in \mathcal{H}_{\mathrm{reg}}$, $H = H_0 + h$ is non-degenerate, and $(H,J)$ is regular.
\end{enumerate}
\end{theo}

Therefore, we can guarantee that for any pair $(H,J)$ of non-degenerate Hamiltonian and almost complex structure there is another pair $(H',J')$ arbitrarily close (in the $\mathcal{C}^1$ sense) such that it is regular.

With all these results we are able to construct the differential of the Floer complex. If we define the space of trajectories
\[\mathcal{L}(x,y) = \quocient{\mathcal{M}(x,y)}{\R}\]
(where we are taking the quotient by the translation action in the $s$ coordinate), then we know that it is compact and that
\[\dim(\mathcal{L}(x,y)) = \mu_{CZ}(x)-\mu_{CZ}(y)-1 .\]

Therefore,

\begin{lema} \label{lema:indices}
If $\mu_{CZ}(x) = \mu_{CZ}(y) + 1$, then $\mathcal{L}(x,y)$ is a finite set. Moreover, if $\mu_{CZ}(x) = \mu_{CZ}(y) + 2$, then $\mathcal{L}(x,y)$ is a finite union of compact and connected $1$-dimensional manifolds.
\end{lema}

With this in mind, if we denote by $n(x,y) = \# \mathcal{L}(x,y)$ (when they have consecutive indices), we can define the differential of the Floer complex.

\begin{deff}
The $k$-th differential of the Floer complex $\partial_k : CF_k(M,H,J) \rightarrow CF_{k-1}(M,H,J)$ can be defined over the generators of $CF_k(M,H,J)$ as
\[\partial_kx = \sum_{\mu_{CZ}(y) = \mu_{CZ}(x) -1} n(x,y)y .\]
\end{deff}

As a consequence of \ref{lema:indices}, $\partial_k \circ \partial_{k+1} = 0$.
