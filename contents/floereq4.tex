\section{$\mathcal{M}$ is compact} \label{section:floerq_compact}

In this section we will tackle the first question about $\mathcal{M}$, in particular, its topology. To prove the results in this section and the next one, we will rely in two important results. The first is well known: the Ascoli-Arzelà theorem (in its general version):

\begin{theo} \label{ascoli_arzela}
{\bf Ascoli-Arzelà:} Let $X$ a localy compact metric space, $V$ a finite dimensional vector space, and $F \subset \mathcal{C}(X,V)$. Then, $F$ is relatively compact in $\mathcal{C}_{\text{loc}}(X,V)$ if, and only if, $F$ is equicontinuous and pointwise bounded.

Recall that $F$ is equicontinuous when, $\forall x \in X, \e > 0$ there is a $\delta > 0$ such that, $\forall y \in X$ with $d(x,y) < \delta$ and $\forall f \in F$, $\|f(x)-f(y)\| < \e$. Pointwise boundedness, on the other hand, means that $\forall x \in X \ \exists K$ such that $\|f(x)\| < M \ \forall f \in F$.
\end{theo}

In our case, we will take $V = \R^m$, where $m$ is such that $M$ can be compactly embedded into $V$. 

The other result that we will use is one about the elliptic regularity of the Floer equation, which can be checked in \cite{audin2014morse}:

\begin{prop} \label{floereq_ellipticreg}
If $(u_k)_k$ is a sequence of solutions of the Floer equation and $u_k \xrightarrow[k \rightarrow \infty]{} u_0$ in $\mathcal{C}_{\text{loc}}^0(\R \times \con{S}^1,M)$, then $u_0 \in \mathcal{C}^{\infty}(\R \times \con{S}^1,M)$, $u_0$ is also a solution of the Floer equation, and $u_k \xrightarrow[k \rightarrow \infty]{} u_0$ in the strong sense in $\mathcal{C}^{\infty}(\R \times \con{S}^1,M)$.
\end{prop}

With these results in hand, we are able to prove the following theorem:

\begin{theo} \label{floer_compact}
The set of solutions of the Floer equation with finite energy, $\mathcal{M}$, is compact.
\end{theo}

The whole content of the proof of this fact is in the following proposition:

\begin{prop} \label{floer_compact_prop}
Under the assumption of asphericallity (\ref{assumption1}), there exists some $A > 0$ such that
\[\|\grad_{(s,t)}u\| \leq A \ \ \ \forall u \in \mathcal{M}, \forall (s,t) \in \R \times \con{S}^1 .\]
\end{prop}

We will first prove the theorem using the proposition, and then the proposition itself:

\begin{proof} {\it (Theorem):} By the proposition \ref{floer_compact_prop}, $\mathcal{M}$ is equicontinuous (because the gradients are uniformly bounded). On the other hand, as $M$ is compact, $\|u(s,t)\|$ is uniformly bounded for $u \in \mathcal{M}$. Therefore, by \ref{ascoli_arzela}, $\mathcal{M}$ is relatively compact in $\mathcal{C}_{\text{loc}}^0(\R \times \con{S}^1,M)$.

Let $(u_n)_n \subset \mathcal{M}$ a sequence. As we just observed, it must have some subsequence $(u_{n_k})$ such that $u_{n_k} \xrightarrow[k \rightarrow \infty]{} u_0$ for some $u_0 \in \mathcal{C}^0(\R\times\con{S}^1,M)$ in the $\mathcal{C}_{\text{loc}}^0(\R\times\con{S}^1,M)$ sense. However, by \ref{floereq_ellipticreg} we deduce that $u_0 \in \mathcal{M}$ and that $u_{n_k} \xrightarrow[k\rightarrow \infty]{} u_0$ in the strong sense in $\mathcal{C}^{\infty}(\R \times \con{S}^1,M)$.
\end{proof}

In order to prove the proposition \ref{floer_compact_prop} we are going to need the following lemma:

\begin{lema}
{\it (Half-max lemma):} Let $X$ be a complete metric space, and $g : X \rightarrow  [0,+\infty)$ a continuous function. Let $x_0 \in X, \e_0 > 0$. Then, there exist $y \in X$ and $\e \in (0,\e_0]$ such that
\begin{enumerate}
	\item $d(y,x_0) \leq 2\e_0$.
	\item $\e_0 g(x_0) \leq \e g(y)$.
	\item $g(x) \leq 2g(y) \ \forall x \in B(y,\e)$.
\end{enumerate}
\end{lema}

\begin{proof} {\it (Proposition \ref{floer_compact_prop}):} First of all, for the purposes of this proof, let us regard the elements of $\mathcal{M}$ as functions $u : \R^2 \rightarrow M$ that are periodic in the second component.

We will prove the proposition by contradiction, so we suppose that there are sequences $(u_k)_k \subset \mathcal{M}$, $((s_k,t_k))_k \subset \R$ such that
\[\lim_{k\rightarrow \infty} \|\grad_{(s_k,t_k)}u_k\| = +\infty .\]
Under this assumption, we can choose a sequence $(\e_k)_k \subset (0,+\infty)$ such that
\[\lim_{k\rightarrow \infty} \e_k \|\grad_{(s_k,t_k)}u_k\| = +\infty .\]
(For instance, we can take the sequence $\e_k = \|\grad_{(s_k,t_k)}u_k\|^{-\frac12}$).
Let $g_k = \|\grad u_k\|$. Then, we can apply the half-max lemma for each $k$, taking $x_0 = (s_k,t_k)$ and $\e_0 = \e_k$. Then, there exist sequences $\e_k'$, $(s_k',t_k')$ such that
\begin{itemize}
	\item $\displaystyle\lim_{k\rightarrow \infty}\e_k'\|\grad_{(s_k',t_k')}u_k\| = +\infty$.
	\item $2\|\grad_{(s_k',t_k')}u_k\| \geq \|\grad_{(s,t)}u_k\| \ \forall (s,t) \in B((s_k',t_k'),\e_k')$.
\end{itemize}

Take $R_k = \|\grad_{(s_k',t_k')}u_k\|$, and consider the sequence of functions $(v_k)_k$ defined by
\[v_k(s,t) = u_k \left(\frac{s}{R_k}+s_k', \frac{t}{R_k}+t_k' \right) .\]
In this case,
\[\grad_{(s,t)}v_k = \frac1{R_k} \grad_{\left( \frac{s}{R_k} + s_k', \frac{t}{R_k} + t_k' \right)} u_k ,\]
so, by definition, $\|\grad_{(0,0)}v_k\|=1$ for all $k$. Moreover, for all $(s,t) \in B(0,\e_k'R_k)$ we have that
\[\|\grad_{(s,t)}v_k\| = \frac1{R_k} \left\|\grad_{\left( \frac{s}{R_k}+s_k', \frac{t}{R_k} + t_k' \right)}u_k \right\| \leq \frac1{R_k} 2R_k = 2 \Rightarrow \]
\[\Rightarrow \|\grad_{(s,t)}v_k\| \leq 2 .\]
As $(u_k)$ are solutions of the Floer equation, we know that
\[\frac{\partial v_k}{\partial s} + J_{v_k} \frac{\partial v_k}{\partial t} + \frac1{R_k} \grad_{\frac{t}{R_k} + t_k'}H(v_k) = 0 .\]
Given these conditions, we know that $(v_k)_k$ is a pointwise bounded, equicontinuous family. Therefore, by theorem \ref{ascoli_arzela}, we know that there is a (sub)sequence (which we will denote by $(v_k)_k$ in an abuse of notation) that has a limit $v \in \mathcal{C}_{\text{loc}}^0(\R^2,M)$. Moreover, if we use the proposition \ref{floereq_ellipticreg} we conclude that $v \in \mathcal{C}^{\infty}(\R^2,M)$ and
\[\frac{\partial v}{\partial s} + J \frac{\partial v}{\partial t} = 0,\]
so $v$ is $J$-holomorphic. In addition,
\[\|\grad_{(0,0)}v\| = 1 ,\]
and
\[\|\grad_{(s,v)}v\| \leq 2 \forall (s,t) \in \R^2,\]
because the balls $B(0,\e_k'R_k)$ tend to cover the entire plane as $k \rightarrow \infty$, and $\|\grad_{(s,t)}v_k\|$ is bounded by $2$ in these balls.

Now we will use this limit $v_k \rightarrow v$ to construct something forbidden by our asphericallity assumption: a sphere with nonzero symplectic area. To do this, we need to begin by checking that $v$ has finite energy.

Consider $B_k = B((s_k',t_k'),\e_k')$. We have that
\[\int_{B(0,\e_k'R_k)} \|\grad v_k\|^2 = \int_{B_k} \|\grad u_k\| dt ds = \int_{B_k} \left( \left\| \frac{\partial u_k}{\partial s} \right\|^2 + \left\| \frac{\partial u_k}{\partial t} - X_t(u_k) + X_t(u_k) \right\|^2 \right)dt ds \leq\]
\[\leq \int_{B_k} \left( \left\|\frac{\partial u_k}{\partial s}\right\|^2 + \left\|\frac{\partial u_k}{\partial t} - X_t(u_k) \right\|^2 + \|X_t(u_k)\|^2 \right) dt ds .\]
Now let us study separately each of the terms. By the corollary \ref{coro:floer_energy_bound} (that we will prove in the next section) we know that the energy of $u_k$ is bounded as $k \rightarrow \infty$, so
\[\int_{B_k} \left( \left\| \frac{\partial u_k}{\partial s} \right\|^2 + \left\| \frac{\partial u_k}{\partial t} - X_t(u_k) \right\|^2 \right) ds dt \leq 2E(u_k) \leq 2C,\]
for some $C > 0$. On the other hand,
\[\int_{B_k} \|X_t(u_k)\|^2 \leq |B_k| \sup_{p \in M} \|X_t(p)\|^2 < +\infty ,\]
%TODO: Reference for Fatou's lemma?
because $M$ is compact and $|B_k| \xrightarrow[k \rightarrow \infty]{} 0$. Therefore, by Fatou's Lemma, $E(v) < +\infty$.

Now, we are ready to see that the symplectic area of $v$ is finite and nonzero:
\[\int_{\R^2} v^{\ast} \omega = \int_{\R^2} \omega\left(\frac{\partial v}{\partial s}, \frac{\partial v}{\partial t}\right) = \int_{\R^2} \omega\left(-J_v \frac{\partial v}{\partial t}, \frac{\partial v}{\partial t}\right) = \int_{\R^2} \left\| \frac{\partial v}{\partial t} \right\|^2 < + \infty ,\]
because the energy is finite (as we just saw). Moreover, the last integral has to be nonzero because $\|\grad_{(0,0)}v\| = 1$.

Moreover, we claim that there exists a sequence $r_k \rightarrow \infty$ such that the length of $v(\partial B(0,r_k))$ tends to $0$ as $k \rightarrow \infty$. To prove this, as $v^{\ast} \omega$ is a symplectic form in $\R^2$, we may express it in polar coordinates. Thus, there is a function $f : [0,+\infty) \times [0,2\pi] \rightarrow (0,+\infty)$ such that
\[v^{\ast} \omega(\rho,\theta) = f(\rho,\theta) \rho \dd \theta \wedge \dd \rho .\]
This induces the Riemannian metric in $\R^2$ defined by $f(\rho,\theta)(\dd\rho^2 + \rho^2 \dd\theta^2)$. Thus, the length of $v(\partial B(0,r))$ is
\[l(r) = r \int_0^{2\pi} \sqrt{f(r,\theta)} d \theta .\]
On the other hand, the area function is
\[A(r) = \int_{B(0,r)} v^{\ast} \omega = \int_0^{2\pi} \left( \int_0^r f(\rho,\theta) d\rho \right) d\rho .\]
As we just proved, $A(r)$ is a bounded function. Its derivative has the expression
\[A'(r) = r \int_0^{2\pi} f(r,\theta) d\theta .\]
Finally, if we apply the Cauchy-Schwarz inequality to the integral in the definition of $l(r)$, we find that
\[l(r) \leq r \sqrt{\int_0^{2\pi} d\theta \int_0^{2\pi} f(r,\theta) d\theta} = r \sqrt{2\pi \frac{A'(r)}{r}}  \Rightarrow\]
\[\Rightarrow l^2(r) \leq 2\pi r A'(r) .\]
Therefore, as $A(r)$ is bounded, there exists a sequence $(r_k)_k$ such that $\displaystyle\lim_{k\rightarrow \infty} r_k A'(r_k) = 0$. To prove this, for instance, we know that
\[\lim_{k \rightarrow \infty} \frac{A(k^2) - A(k)}{\ln k} = 0 ,\]
and
\[\frac{A(k^2)-A(k)}{\ln k} = \frac{A(k^2)-A(k)}{\ln k^2 - \ln k} = \frac{A'(r_k)}{\quocient{1}{r_k}}\]
for some $k \leq r_k \leq k^2$, by the mean value theorem.

With all the facts that we have proved, we are in the situation to explain the main phenomenon: $v$ forms a "bubble" inside of $M$, to the point that it tends towards a sphere.

Let $k$ sufficiently large so that $\gamma = v(\partial B(0,r_k))$ is contained inside a Darboux chart for $\omega$, $U \subset M$. Inside of $U$, we have that $\omega = d\lambda$ for some $1$-form $\lambda$. As $\gamma \subset U$, we can take $D_k$ a closed disk with boundary $\gamma$ inside of $U$. Therefore, $v(B(0,r_k)) \cup D_k$ is a sphere $S_{r_k}^2 \subset M$. By the assumption of asphericallity, we know that
\[\int_{S_{r_k}^2} i^{\ast} \omega = 0 ,\]
so
\[\int_{D_k} i^{\ast} \omega + \int_{v(B(0,r_k))} \omega = 0 .\]
The first term tends to $0$ as $k$ goes to infinity, because
\[\int_{D_k} i^{\ast} \omega = \int_{D_k} d\lambda = \int_{v(\partial B(0,r_k))} \lambda ,\]
and thus
\[\left| \int_{D_k} \omega \right| \leq \text{length}(\gamma) \sup_U\|\lambda\| \xrightarrow[k \rightarrow \infty]{} 0 .\]
On the other hand, as $r_k \xrightarrow[k \rightarrow \infty]{} \infty$, the second term satisfies that
\[\int_{v(B(0,r_k))} \omega = \int_{B(0,r_k)} v^{\ast} \omega \xrightarrow[k \rightarrow \infty]{} \int_{\R^2} v^{\ast} \omega > 0 .\]
Therefore,
\[\lim_{k \rightarrow \infty} \int_{S_{r_k}^2} \omega > 0,\]
which contradicts our assumption. Therefore, our assumption in this proof must be wrong: there has to be a bound $A$ such that
\[\|\grad_{(s,t)}u\| \leq A \ \ \ \forall u \in \mathcal{M}, (s,t) \in \R \times \con{S}^1 .\]
\end{proof}
