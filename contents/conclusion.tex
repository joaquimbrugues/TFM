The first main result derived from the Floer homology is the one it was intended for in the first place:

\begin{rmrk}
Let $M$ be a compact, symplectic manifold. Suppose that it is aspherical, this means, that it satisfies the assumptions \ref{assumption1} and \ref{assumption2}. Then, the conjecture \ref{theo:arnold_conjecture} is true: for any time-dependent Hamiltonian $H_t$, the number of non-degenerate periodic orbits of the flow $X_H$ is greater or equal than
\[\sum_k \mathrm{dim}(H_k(M)) .\]
\end{rmrk}

Nonetheless, Floer homology is a versatile tool that plays an important role in modern symplectic topology. Its implications are still being studied.

%TODO: Dir que hem fet.
