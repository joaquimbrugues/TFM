\section{The conditions on the manifold}

Now, as a prelude to the definition of the Floer complex in full rigour, we follow through the essential assumptions that we need to do in order to work. The reasons why these constrictions are set upon the topology of $M$ will be clear as we define the Floer complex.

The assumptions that we are using are those of asphericality, this means,

\begin{assump} \label{assumption1}
For every smooth map $\psi : \con{S}^2 \rightarrow M$ we have that
$$\int_{\con{S}^2} \psi^{\ast} \omega = 0 .$$
\end{assump}

This condition can be understood as that $\omega$ is zero over spheres inside the manifold.

\begin{assump} \label{assumption2}
For every smooth map $\psi : \con{S}^2 \rightarrow M$ the fiber bundle $\psi^{\ast} TM$ admits a symplectic trivialization.
\end{assump}

These two assumptions are met if we assume the (more restrictive) condition that $\pi_2(M) = 0$, this means, the second homotopy group of $M$ is trivial.
