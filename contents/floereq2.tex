\section{The action functional}

Now that we have defined the space of interest, this means, $\mathcal{L}M$, we need a way to emulate the constructions in Morse theory. In Morse theory we were studying a Morse function $f$ on $M$, graduated by its index and connecting the critical points via a pseudogradient adapted to $f$. Here, we want to provide a smooth functional\footnote{We call {\it functionals} the functions defined in a space of infinite dimensions} from $\mathcal{L}M$ to $\R$ such that its critical points are the 1-periodic solutions of $X_{H_t}$. This will be precisely the action functional.

\begin{rmrk}
Take $x \in \mathcal{L}M$ a contractible loop, and let $u : \con{D}^2 \rightarrow M$ a smooth map such that $\left. u \right|_{\con{S}^1} = x$. Then, we can define the quantity
\[\int_{\con{D}^2} u^{\ast}\omega .\]
Assumption \ref{assumption1} guarantees that this number does not depend on the extension $u$ that we choose for $x$, but only on the loop. Indeed, if we take two extensions $u, \tilde{u} : \con{D}^2 \rightarrow M$, we can glue them along the boundary, so we get a continuous map $\psi : \con{S}^2 \rightarrow M$. Then,
\[\int_{\con{D}^2} u^{\ast} \omega - \int_{\con{D}^2} \tilde{u}^{\ast} \omega = \int_{\con{S}^2} \psi^{\ast} \omega = 0 .\]
\end{rmrk}

With this remark, we can proceed to define the action functional:

\begin{deff}
Let $H_t$ a 1-periodic Hamiltonian. The {\bf action functional} is the map $\mathcal{A}_H : \mathcal{L}M \rightarrow \R$ defined by
\[\mathcal{A}_H(x) = \int_0^1 H_t(x(t)) d t - \int_{\con{D}^2} u^{\ast} \omega ,\]
where $u$ is any extension of $x$ to the disk.
\end{deff}

As we just saw, this is well defined, because it does not depend on $u$.

For instance, if we take $M = \R^{2n}$ (not compact, but good enough for the present example) with the symplectic structure $\omega = \sum d p_i \wedge d q_i$, we know that $\omega = d \alpha$, with $\alpha = \sum p_i d q_i$, so, for any $u : \con{D}^2 \rightarrow \R^{2n}$ extending the loop $x$, we have that
\[\int_{\con{D}^2} u^{\ast} \omega = \int_{\con{D}^2}  u^{\ast} d \alpha = \int_{\con{S}^1} x^{\ast} \alpha .\]

Thus, the action functional is
\[\mathcal{A}_H (x) = \int_{\con{S}^1} \left( H_t d t - p d q \right) .\]

Going back to the general case, let us state the lemma that underlines the importance of the action functional for our purposes, this means, studying the 1-periodic solutions of $X_H$:

\begin{prop}
A loop $x \in \mathcal{L}M$ is a critical point of $\mathcal{A}_H$ if, and only if, $x$ is a 1-periodic orbit of $X_H$.
\end{prop}

\begin{proof}
Let us compute the differential of $\mathcal{A}_H$ at $x$ along $Y \in T_x\mathcal{L}M$. We need to extend $x$ to a curve through $x$. More precisely, we need to define $z : (-\e,\e) \times \con{S}^1 \rightarrow M$ such that

\begin{itemize}
	\item $z(0,t) = x(t)$.
	\item $\left. \frac{d}{d s} \right|_{s=0} z(s,t) = Y(t)$.
\end{itemize}

Moreover, we need to take $u : \con{D}^2 \rightarrow M$ an extension of $x$ to the disk, and take $\tilde{u} : (-\e,\e) \times \con{D}^2 \rightarrow M$ such that

\begin{itemize}
	\item $\tilde{u}(s,e^{it}) = z(s,e^{it})$.
	\item $\tilde{u}(0,p) = u(p)$.
\end{itemize}

We can then extend $Y$ to $\con{D}^2$ by setting
\[Y(p) = \frac{\partial \tilde{u}}{\partial s} (0,p) .\]
Then,
\[d \mathcal{A}_H(x) \cdot Y = \left. \frac{d}{d s} \right|_{s=0} \mathcal{A}_H(z(s)) = \left. \frac{d}{d s} \right|_{s=0} \left( \int_0^1 H_t(z(s,t) d t - \int_{\con{D}^2} \tilde{u}_s^{\ast} \omega \right) .\]
Differentiating the second term we get
\[- \int_{\con{D}^2} \left( \left. \frac{d}{d s} \right|_{s=0} \tilde{u}_s^{\ast} \omega \right) = - \int_{\con{D}^2} u^{\ast} \left( \mathcal{L}_{Y(p)} \omega \right) =\]
(applying Cartan's formula, and then Stoke's theorem)
\[= - \int_{\con{D}^2} u^{\ast} \left( d i_{Y(p)} \omega \right) = - \int_{\con{S}^1} x^{\ast} \left( i_{Y(t)} \omega \right) = \int_0^1 \omega(\dot{x}(t),Y(t)) d t .\]

On the other hand, if we differentiate the first term,
\[\int_0^1 \left. \frac{\partial}{\partial s} \right|_{s=0} H_t(z(s,t)) d t = \int_0^1 d H_t(x(t)) \cdot Y(t) d t ,\]
and, by the definition of $X_H$, this is equal to
\[- \int_0^1 \omega(X_{H_t},Y(t)) d t .\]
Therefore,
\[d \mathcal{A}_H(x) \cdot Y = \int_0^1 \omega(\dot{x}(t) - X_{H_t}(x(t)), Y(t)) d t .\]

Then, $x$ is a critical point of $\mathcal{A}_H$ if and only if this expression is $0$ for all $Y \in T_x\mathcal{L}M$ and, by the non-degeneracy of $\omega$, this happens if and only if $\dot{x}(t) = X_{H_t}(x(t))$, so it is a periodic solution to the system.
\end{proof}
