\documentclass{beamer}
\usepackage[utf8]{inputenc}
\usepackage[catalan]{babel}
\usepackage{amsfonts}
\usepackage{amsmath}
\usepackage{amssymb}
\usepackage{tikz}
\usepackage{float}
\usepackage{pgfplots}
\usetikzlibrary{arrows.meta}
\usetikzlibrary{decorations.markings}
\usepackage{graphicx}

\newcommand{\con}[1]{\mathbb{#1}}
\newcommand{\R}{\con{R}}
\newcommand{\C}{\con{C}}
\newcommand{\Z}{\con{Z}}
\newcommand{\e}{\varepsilon}
\newcommand{\de}{\delta}
\newcommand{\quocient}[2]{{\raisebox{.1em}{$#1$}\left/\raisebox{-.1em}{$#2$}\right.}}

\newtheorem{teorema}{Teorema}

\usetheme{Hannover}

%gets rid of bottom navigation symbols
\setbeamertemplate{navigation symbols}{}

\title{Homologia de Floer}

\author{Joaquim Brugués Mora}

\institute{Facultat de Matemàtiques i Estadística\\ UPC}

\date{\today}

\makeatletter
	\setbeamertemplate{sidebar \beamer@sidebarside}%{sidebar theme}
	{
		\beamer@tempdim=\beamer@sidebarwidth%
		\advance\beamer@tempdim by -6pt%
		\insertverticalnavigation{\beamer@sidebarwidth}%
		\vfill
		\ifx\beamer@sidebarside\beamer@lefttext%
		\else%
			\usebeamercolor{normal text}%
			\llap{\usebeamertemplate***{navigation symbols}\hskip0.1cm}%
			\vskip2pt%
		\fi%
	}%
\makeatother

\begin{document}

\begin{frame}
	\titlepage
\end{frame}

\begin{frame}{Índex}
	\tableofcontents
\end{frame}

\section{Introducció}

\begin{frame}{Teoria de Morse}
	\begin{block}{Idea}
		L'homologia de Morse relaciona la topologia d'una varietat $M$ amb els punts crítics d'una funció de Morse $f: M \rightarrow \R$.
	\end{block}

	\begin{block}{Definició}
		El {\bf complex de Morse} d'una varietat $M$ amb funció de Morse $f$ és el complex amb grup $CM_k(M,f)$ generat (sobre $\con{Z}_2$) pels punts crítics d'índex $k$ de $f$, i tal que
		\[\partial_k(a) = \sum_{\mathrm{Ind}(b) = k-1} n(a,b) b ,\]
		on $\mathrm{Ind}(a) = k$, i $n(a,b)$ és el nombre de trajectories de $-\mathrm{grad}f$ que uneixen $a$ amb $b$ (mòdul 2).
	\end{block}
\end{frame}

\begin{frame}{Desigualtats de Morse}
	L'homologia de Morse $HM_{\bullet}(M)$ induïda pel complex de Morse no depén de la funció $f$ ni de la mètrica $g$ utilitzades per definir-la sinó que és un invariant topològic de la varietat $M$.

	\begin{teorema}[Desigualtats de Morse]
		Sigui $c_k$ el nombre de punts crítics d'índex $k$ d'una funció de Morse, i sigui $\beta_k$ el $k$-èssim nombre de Betti de $M$. Aleshores,
		\[c_k \geq \beta_k ,\]
		és a dir,
		\[\# \mathrm{Crit}(f) \geq \sum_{k=0}^n \beta_k .\]
	\end{teorema}
\end{frame}

\section{Motivació}

\begin{frame}{La conjectura d'Arnold}
	Sigui $(M,\omega)$ una varietat simplèctica.

	\begin{teorema}
		Sigui $H_t : \R \times M \rightarrow \R$ un Hamilonià depenent del temps. Llavors, si $N$ és el nombre d'òrbites periòdiques de $X_{H_t}$,
		\[N \geq \sum_{k=0}^{2n} \beta_k .\]
	\end{teorema}

	\begin{block}{Observació}
		En el cas d'un Hamiltonià $H$ no depenent del temps el teorema és cert com a conseqüència de les desigualtats de Morse.
	\end{block}
\end{frame}

\begin{frame}{Premisses}
	\begin{block}{Esquema}
		La idea és construïr un complex de cadenes com en el cas de Morse, generat per les solucions periòdiques de $X_{H_t}$.

		Però per fer-ho hem de posar condicions sobre el problema.
	\end{block}

	\begin{itemize}
		\item {\bf Asfericitat:} Qualsevol esfera dins de $M$ es pot contraure a un punt (malgrat que es poden imposar condicions més febles sobre $M$).
		\item {\bf Llaços contràctils:} Només considerem els llaços $x : \con{S}^1 \rightarrow M$ que es poden contraure a un punt.
	\end{itemize}
\end{frame}

\section{El complex de Floer}

\begin{frame}{El complex de Floer}
	\begin{itemize}
		\item El complex està generat per les solucions 1-periòdiques i contràctils de
		\[\dot{x} = X_{H_t}(x) .\]
		La condició de regularitat, en aquest cas, és que $\mathrm{det}(d_{x(0)}\varphi_{X_t}^1 - \mathrm{Id}) \neq 0$.
		\item Classifiquem les òrbites periòdiques amb {\bf l'índex de Conley-Zehnder}
		\[\mu_{CZ} : SP(M) \longrightarrow \con{Z} .\]
		\item La connexió entre òrbites periòdiques es construeix amb les solucions d'energia finita de l'{\bf equació de Floer}
		\[\frac{\partial u}{\partial s} + J \frac{\partial u}{\partial t} + \mathrm{grad}_uH_t = 0 .\]
	\end{itemize}
\end{frame}

\begin{frame}{L'equació de Floer (I)}
	El {\bf funcional d'acció} és
	\[\mathcal{A}_H(x) = \int_0^1 H_t(x(t)) dt - \int_{\con{D}^2} u^{\ast} \omega .\]

	\begin{block}{Proposició}
		Un llaç $x$ és un punt crític de $\mathcal{A}_H$ si i només si $\dot{x} = X_{H_t}(x)$.
	\end{block}
\end{frame}

\begin{frame}{L'equació de Floer (II)}
	\begin{block}{Definició}
		Una {\bf estructura quasi complexa} $J$ en $M$ és una secció de $TM\otimes TM^{\ast}$ tal que
		\[J^2 = - \mathrm{Id} .\]
		Una estructura quasi complexa i ina estructura simplèctica indueixen una mètrica Riemanniana $g$ en $M$.
	\end{block}

	En aquest cas, el {\bf gradient de} $\mathcal{A}_H$ és
	\[\mathrm{grad}_x\mathcal{A}_H = J \dot{x} + \mathrm{grad}_{x(t)}H_t .\]
	Les trajectòries del flux de $- \mathrm{grad}_x\mathcal{A}_H$ són les funcions $u : \R \times \con{S}^1 \rightarrow M$ tals que
	\[\frac{\partial u}{\partial s} + J \frac{\partial u}{\partial t} + \mathrm{grad}_uH_t = 0 .\]
\end{frame}

\begin{frame}{L'equació de Floer (III)}
	\begin{itemize}
		\item Les solucions estacionàries són les òrbites periòdiques de les equacions de Hamilton de $H_t$,
		\[\dot{x} = X_{H_t}(x) .\]
		\item Les solucions amb $\frac{\partial u}{\partial t} = 0$ són les trajectòries del gradient negatiu,
		\[\frac{\partial u}{\partial s} = - \mathrm{grad}_uH_t ,\]
		i connecten punts crítics (en el cas que $H$ no depengui del temps).
		\item Si $\mathrm{grad}_uH_t = 0$, llavors l'equació de Floer és l'equació de Cauchy-Riemann
		\[\frac{\partial u}{\partial s} + J \frac{\partial u}{\partial t} = 0 .\]
	\end{itemize}
\end{frame}

%\begin{frame}{The Floer equation (IV)}
%	To guarantee that the solutions connect critical points of $\mathcal{A}_H$ we need to add the condition that
%	\[E(u) < + \infty\]

%	\begin{block}{Definition}
%		The {\bf energy} of a solution of the Floer equation is
%		\[E(u) = \int_{\R \times \con{S}^1} \left| \frac{\partial u}{\partial s} \right|^2 dsdt = \int_{\R \times \con{S}^1} \left| \frac{\partial u}{\partial t} - X_{H_t} \right|^2 dsdt\]
%	\end{block}
%
%	\begin{block}{Remark}
%		We have that $E(u) = 0$ if and only if $u(s,t) = x(t)$ and it is a periodic solution.
%	\end{block}
%\end{frame}

\begin{frame}{Resultats}
	\begin{teorema}
		L'homologia de Floer $HF_{\bullet}(M)$ és un {\bf invariant topològic}: no depén de $H_t$ ni de $J$.
	\end{teorema}

	\begin{teorema}
		L'homologia de Floer d'una varietat és isomòrfica a l'homologia de Morse (amb un desplaçament de l'índex),
		\[HF_{\bullet}(M) \cong HM_{\bullet+n}(M) .\]
	\end{teorema}
\end{frame}

\section{Conclusions}

\begin{frame}{Conclusions}
	\begin{itemize}
		\item La conjectura d'Arnold queda demostrada fent servir l'homologia de Floer en el cas de varietats amb $\pi_2(M) = 1$.
		\item A més, aquesta tècnica obre el camí per a la demostració en casos més generals.
		\item L'esquema utilitzat per construir l'homologia de Morse és el mateix que s'utilitza per construir l'homologia de Floer.
	\end{itemize}
\end{frame}

\section{Possibles extensions}

\begin{frame}{Extensions}
	Hi ha diverses vies en les quals es pot estendre la teoria de Floer:
	\begin{itemize}
		\item Estendre la definició del complex de Floer a varietats més generals (b-simplèctiques, per començar).
		\item Estendre la construcció a varitats no asfèriques.
		\item Estudiar òrbites periòdiques no contràctils.
	\end{itemize}
\end{frame}

\begin{frame}
	\begin{center}
		\huge Gràcies per la vostra atenció!
	\end{center}
\end{frame}

\end{document}
