\section{The Morse complex}

With all the concepts introduced in the last section, we are able to construct the Morse complex.

\begin{deff}
Recall that a {\bf complex} over a ring $R$ is a sequence of modules $\{ C_k \}_{k \in \N}$ over $R$ and a sequence of morphisms $\partial_k : C_k \rightarrow C_{k-1}$ such that $\partial_{k-1} \circ \partial_k = 0$.
\end{deff}

In the case of the complexes defined in this thesis, the ring that we are taking is $\con{Z}_2$. In this section we are defining the modules of the Morse complex, and the differential (which is the usual name for the maps $\partial_{\bullet}$) will be defined properly later.

\begin{deff}
The $k$-th group of the Morse complex of the manifold $M$ with the function $f$ is the free module over $\con{Z}_2$ generated by the critical points of index $k$ of the function $f$:

\begin{displaymath}
C_k(M,f) := \langle \crit_k(f) \rangle_{\con{Z}_2} .
\end{displaymath}
\end{deff}

\begin{exmpl}
Let $S^2 \subset \R^3$ be the 2-sphere seen as a submanifold of $\R^3$, and consider the height function $h(x,y,z) = z$ in $\R^3$, but restricted to $S^2$. Then, it can be checked easily that the only critical points of $h$ are the north and south poles, $N = (0,0,1)$ and $S = (0,0,-1)$, and that the index of $h$ is respectivelly $2$ and $0$. Therefore, the Morse complex is isomorphic to

$$C_k(S^2,h) \cong \left\{ \begin{array}{lc} \con{Z}_2 & \text{if } k = 0, 2 \\ 0 & \text{otherwise} \end{array}\right. .$$
\end{exmpl}
