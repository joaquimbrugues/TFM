\section{Limit endpoints of elements of $\mathcal{M}$}

We devote this section to prove what we already hinted in the remark \ref{remark:floer_limits}: the elements of $\mathcal{M}$ connect the critical points of the action functional. The precise statement of the theorem is the following:

\begin{theo} \label{theo:floer_endpoints}
If all the trajectories of $X_t$ are non-degenerate, and under the assumption of asphericallity, then for all $u \in \mathcal{M}$ there are $x, y \in \mathcal{L}M$ critical points of $\mathcal{A}_H$ such that
\[\lim_{s\rightarrow - \infty} u(s,\cdot) = x \ , \ \ \ \lim_{s\rightarrow +\infty} u(s,\cdot) = y\]
in the $\mathcal{C}^{\infty}(\con{S}^1,M)$ sense, and
\[\lim_{s\rightarrow \pm\infty} \frac{\partial u}{\partial s} (s,t) = 0\]
uniformly in $t$.
\end{theo}

If we prove this theorem, we will have constructed the analogous as the pseudogradient adapted to a function in a manifold $M$ in Morse theory: a flow that connects critical points in a meaningful way. In the Morse case, this flow appears integrating the pseudogradient flow, but in the Floer case we needed to be a little more subtle and avoid the proper definition of the vector field, defining instead the equation that the flow must satisfy.

First of all, we need a basic lemma on what the nondegeneracy of $X_t$ implies for the critical points of the action functional:

\begin{lema} \label{lema:floer_finite}
Under the assumption of nondegeneracy, the number of critical points of $\mathcal{A}_H$ (or the number of periodic orbits of $X_t$) is finite.
\end{lema}

\begin{proof}
Let us consider the following submanifolds of $M\times M$:
\begin{itemize}
	\item The diagonal $\Delta = \{(x,x) \ | \ x \in M\}$.
	\item The graph of the flow $\varphi_{X_t}^t$ of $X_t$ at time $1$: $F = \left\{\left(x,\varphi_{X_t}^1(x)\right) \ | \ x \in M\right\}$.
\end{itemize}
The fact that the periodic orbits of $X_t$ are nondegenerate is equivalent to $\Delta \pitchfork F$. Moreover, $\dim(\Delta) = \dim(F) = 2n$ and $\dim(M\times M) = 4n$, so $\dim(\Delta \cap F) = 0$, so it is a closed submanifold of dimension $0$ of a compact manifold, so it is finite.
\end{proof}

In order to prove the theorem, we need to begin by proving a certain bound in the energy and the action functional, depending on the critical points of $\mathcal{A}_H$. Let us denote $u_s = u(s,\cdot) \in \mathcal{L}M$.

\begin{prop} \label{prop:floer_action_bound}
Let $u \in \mathcal{M}$. There exist $x,y \in \crit(\mathcal{A}_H)$ such that
\[\lim_{s \rightarrow -\infty} \mathcal{A}_H(u_s) = \mathcal{A}_H(x), \ \ \ \lim_{s \rightarrow +\infty} \mathcal{A}_H(u_s) = \mathcal{A}_H(y) .\]
\end{prop}

This has some immediate consequences:

\begin{coro}
If $\mathcal{M}$ is nonempty, then $\crit(\mathcal{A}_H)$ is nonempty.
\end{coro}

This does not even require a proof, as it is an immediate consequence of the proposition.

\begin{coro} \label{coro:floer_energy_bound}
There exist $C_1, C_2 > 0$ such that
\[|\mathcal{A}_H(u)| < C_1 , E(u) < C_2 \ \ \ \ \forall u \in \mathcal{M} .\]
\end{coro}

\begin{proof}
Because of the lemma \ref{lema:floer_finite}, we know that $\crit(\mathcal{A}_H)$ is finite. Therefore, $\mathcal{A}_H(\crit(\mathcal{A}_H))$ is a bounded set in $\R$. Moreover, the function $s \mapsto \mathcal{A}_H(u_s)$ is decreasing, because
\[\frac{d}{ds} \mathcal{A}_H(u_s) = \dd_{u_s} \mathcal{A}_H \cdot \frac{\partial u}{\partial s} = g\left( \grad_{u_s} \mathcal{A}_H, \frac{\partial u}{\partial s} \right) = - \left\|\frac{\partial u}{\partial s}\right\|^2 \leq 0 ,\]
(because $\grad_{u_s}\mathcal{A}_H = - \frac{\partial u}{\partial s}$).

Therefore,
\[\mathcal{A}_H(y) = \lim_{s \rightarrow +\infty} \mathcal{A}_H(u_s) \leq \mathcal{A}_H(u_s) \leq \lim_{s \rightarrow -\infty} \mathcal{A}_H(u_s) = \mathcal{A}_H(x) ,\]
from which follows the bound for $\mathcal{A}_H$. To deduce that the energy must be bounded in $\mathcal{M}$, it is enough to consider that
\[E(u) = \mathcal{A}_H(x) - \mathcal{A}_H(y)\]
because of the remark \ref{remark:floer_limits}, so the energy must also be bounded.
\end{proof}

\begin{proof} {\it (Proposition \ref{prop:floer_action_bound}):} Let us prove the case for $s \rightarrow +\infty$, as the case for $-\infty$ is analogous.

As we just showed, the function $s \mapsto \mathcal{A}_H(u_s)$ is decreasing. As it is also continuous, it is enough to prove that there exists some sequence $(s_k)_k$ with $s_k \xrightarrow[k \rightarrow \infty]{} +\infty$ and some $y \in \crit(\mathcal{A}_H)$ with
\[\lim_{k \rightarrow \infty} \mathcal{A}_H(u_{s_k}) = \mathcal{A}_H(y) .\]
We will prove this in 3 steps:
\begin{enumerate}
	\item There exists a $(s_k)_k$ tending to infinity such that $u_{s_k} \xrightarrow[k\rightarrow \infty]{} y$ in the $\mathcal{C}^0(\con{S}^1,M)$ topology for some $y \in \mathcal{C}^0(\con{S}^1,M)$.
	\item $y$ is of class $\mathcal{C}^{\infty}$, and $y \in \crit(\mathcal{A}_H)$.
	\item $\mathcal{A}_H(u_{s_k}) \xrightarrow[k \rightarrow \infty]{} \mathcal{A}_H(y)$.
\end{enumerate}

{\bf Step 1:} Let $u \in \mathcal{M}$. Its energy is finite, so
\[\int_{-\infty}^{+\infty} \left( \int_0^1 \left| \frac{\partial u}{\partial t} - X_t(u) \right|^2 dt \right) ds < \infty .\]
Therefore, there exists some sequence $(s_k)_k$ with $s_k \xrightarrow[k\rightarrow \infty]{} +\infty$ such that
\[\lim_{k\rightarrow \infty} \left\| \frac{\partial u}{\partial t}(s_k,\cdot) - X_t(u(s_k,\cdot)) \right\|_{L^2}^2 = 0 .\]
Let $u_k = u(s_k,\cdot)$. Taking an embedding of $M$ into $\R^m$, we can consider that
\[\lim_{k\rightarrow \infty} \left\| \dot{u}_k - X_t(u_k) \right\|_{L^2(\con{S}^1,\R^m)}^2 = 0 .\]
$M$ is compact, so $X_t$ is bounded, so there is some $R > 0$ such that $\displaystyle\sup_{p \in M, t \in \con{S}^1} \|X_t(p)\| < R$. Therefore, there is some $B > 0$ with $\|\dot{u}_k\|_{L^2} \leq B$ for all $k$. This implies that $(u_k)_k$ is equicontinuous:
\[\|u_k(t_1) - u_k(t_0)\| = \left\| \int_{t_0}^{t_1} \dot{u}_k(t)dt \right\| \leq \|\ind_{[t_0,t_1]}\|_{L^2} \|\dot{u}_k\|_{L^2} = \sqrt{t_1-t_0} B ,\]
(where we applied the Cauchy-Schwarz inequality). Therefore, applying the theorem \ref{ascoli_arzela} (a subsequence of) $u_k$ has a limit $y$ in the $\mathcal{C}^0(\con{S}^1,\R^m)$ topology or, equivalently, in the $\mathcal{C}^0(\con{S}^1,M)$ topology.

{\bf Step 2:} In order to prove that $y \in \mathcal{C}^{\infty}(\con{S}^1,M)$, we can show that
\[y(t)-y(0) = \int_0^t X_{\tau}(y(\tau)) d\tau .\]
In order to prove this, we need to use the convergence of $u_k$:
\[y(t)-y(0)-\int_0^t X_{\tau}(y(\tau)) d\tau =\]
\[= \lim_{k\rightarrow\infty} \left( u_k(t) - u_k(0) - \int_0^t X_{\tau}(y(\tau)) d\tau \right) = \lim_{k\rightarrow\infty} \left( \int_0^t \dot{u}_k(\tau)d\tau - \int_0^t X_{\tau}(y(\tau)) d\tau \right) =\]
\[= \lim_{k\rightarrow\infty} \left( \int_0^t \left( \dot{u}_k(\tau) - X_{\tau}(u_k(\tau)) \right) d\tau \right) + \lim_{k\rightarrow\infty} \left( \int_0^t \left( X_{\tau}(u_k(\tau)) - X_{\tau}(y(\tau)) \right) d\tau \right) .\]
To prove that the first term converges to zero we can apply the Cauchy-Schwarz inequality and the fact that
\[\left\|\cdot{u}_k - X_t(u_k)\right\|_{L^2} \xrightarrow[k\rightarrow\infty]{} 0.\]
On the other hand, to see that se second term converges to zero too we just need to use the $\mathcal{C}^0$ convergence of $(u_k)_k$, which yields
\[\left| \int_0^t \left( X_{\tau}(u_k(\tau)) - X_{\tau}(y(\tau)) \right) d\tau \right| \leq 2\pi \left\| X_t(u_k) - X_t(y) \right\|_{\mathcal{C}^0} \leq 2\pi R \| u_k - y \|_{\mathcal{C}^0} \xrightarrow[k\rightarrow\infty]{} 0 .\]

Therefore,
\[y(t) = y(0) + \int_0^t X_{\tau}(y(\tau)) d\tau ,\]
so $y$ is differentiable, and
\[\dot{y}(t) = X_t(y(t)) .\]
This allows us to apply a bootstrapping argument, this means, as $y \in \mathcal{C}^0(\con{S}^1,M)$, by this formula it is clear that $y \in \mathcal{C}^1(\con{S}^1,M)$. However, we can apply this argument to show that $y$ is $\mathcal{C}^2$, $\mathcal{C}^3$, and so on, so actually $y \in \mathcal{C}^{\infty}(\con{S}^1,M)$. With the same argument we can prove that $u_k \xrightarrow[k\rightarrow\infty]{} y$ in the $\mathcal{C}^{\infty}$ topology, because all of its derivatives converge uniformly.

{\bf Step 3:} We want to prove that $\mathcal{A}_H(u_k) \xrightarrow[k\rightarrow\infty]{} \mathcal{A}_H(y)$. First of all, as $u_k$ converges uniformly, $(H_t(u_k))_k$ also converges uniformly, so
\[\int_0^1 H_t(u_k(t))dt \xrightarrow[k\rightarrow\infty]{} \int_0^1 H_t(y(t))dt .\]
On the other hand, we need to prove that, if $\widetilde{u_k}$ and $\widetilde{y}$ are extensions to the disk of $u_k$ and $y$ respectively, then
\[\lim_{k\rightarrow\infty} \left( \int_{\con{D}^2} \widetilde{u_k}^{\ast} \omega - \int_{\con{D}^2} \widetilde{y}^{\ast} \omega \right) = 0 .\]
To compute this limit we need to use the local exactness of $\omega$ together with the asphericallity condition (\ref{assumption1}) and the $\mathcal{C}^1$ convergence of $(u_k)_k$. Consider $U \subset M$ an open retractible neighbourhood of $y(\con{S}^1)$, so that $\left.\omega\right|_U$ is exact, so $\left.\omega\right|_U = d\lambda$ for some $\lambda \in \Omega^2(U)$. For $k$ large enough, as $u_k$ converges uniformly, $u_k(\con{S}^1) \subset U$. This way, we can construct a sphere gluing together three surfaces:
\begin{enumerate}
	\item A cylinder $C \subset U$ defining an homotopy between $u_k$ and $y$. We may parametrize it by a smooth map $\varphi : [0,1] \times \con{S}^1 \rightarrow M$ with $\left.\varphi\right|_0 = u_k$ and $\left.\varphi\right|_1 = y$.
	\item The disk $\widetilde{u_k}$ with boundary $u_k$.
	\item The disk $\widetilde{y}$ with boundary $y$.
\end{enumerate}
If we denote by $S$ the sphere obtained by gluing these surfaces by the boundary, and applying the assumption \ref{assumption1}, we know that
\[\int_S i^{\ast} \omega = 0 ,\]
so
\[\int_{\con{D}^2} \widetilde{u_k}^{\ast} \omega - \int_{\con{D}^2} \widetilde{y}^{\ast} \omega = - \int_0^1 \int_{\con{S}^1} \varphi_s^{\ast}\omega ds .\]
Applying the Stokes theorem, and knowing that $\omega$ is exact in $U$, we deduce that
\[\int_0^1 \int_{\con{S}^1} \varphi_s^{\ast}\omega ds = \int_{\con{S}^1} u_k^{\ast} \lambda - \int_{\con{S}^1} y^{\ast} \lambda = \int_{\con{S}^1} \left( \lambda(\dot{u}_k) - \lambda(\dot{y}) \right) dt =\]
\[= \int_{\con{S}^1} \lambda\left( \dot{u}_k - X_t(u_k) \right) dt + \int_{\con{S}^1} \lambda\left( X_t(u_k) - X_t(y) \right) dt ,\]
(where we have applied that $\dot{y} = X_t(y)$).

Let $L = \sup_{p \in C} \|\lambda\|$. Then, the first term of the last expression converges to $0$, because $\dot{u}_k - X_t(u_k)$ converges to $0$. On the other hand,
\[\left| \int_{\con{S}^1}\lambda \left(X_t(u_k) - X_t(y)\right)dt \right| \leq L \|X_t(u_k) - X_t(y)\|_{L^1} \leq L R \|u_k - y\|_{L^1} \xrightarrow[k\rightarrow\infty]{} 0,\]
where the last convergence is due to the fact that $u_k$ converges to $y$ uniformly.
\end{proof}

As the results we have used so far were necessary to prove theorem \ref{floer_compact} we have not alluded to the results in the previous section in our proofs. From now on, however, we can (and will) use the compacity theorem in order to prove theorem \ref{theo:floer_endpoints}.

\begin{rmrk} \label{floer_group_action}
The additive group $(\R,+)$ acts on the right on $\mathcal{M}$, as $(u \cdot \sigma)(s,t) = u(s+\sigma,t)$ for $s,t,\sigma \in \R$.
\end{rmrk}

\begin{lema} \label{lema:floer_sequence_limit}
Let $u \in \mathcal{M}$, and $(s_k)_k$ a sequence of real numbers with $s_k \xrightarrow[k \rightarrow \infty]{} +\infty$. Then, there exists a subsequence $(k_l)_l$ and some $y \in \crit(\mathcal{A}_H)$ such that
\[u \cdot k_l \xrightarrow[l\rightarrow \infty]{} y .\]
\end{lema}

\begin{proof}
Let $u_k = u \cdot s_k$, which is a sequence in $\mathcal{M}$. As this set is compact (by theorem \ref{floer_compact}), there is a subsequence (by abuse of notation we will denote it by $u_k$ as the original sequence) and some element $v \in \mathcal{M}$ such that $u_k \rightarrow v$ in the $\mathcal{C}^{\infty}(\R \times \con{S}^1, M)$ topology. Therefore, for $s \in \R$, and $t \in \con{S}^1$,
\[\lim_{k\rightarrow \infty} u(s+s_k,t) = v(s,t) .\]
However, by proposition \ref{prop:floer_action_bound} there exists a $y \in \crit(\mathcal{A}_H)$ such that
\[\mathcal{A}_H(v(s)) = \lim_{k\rightarrow\infty} \mathcal{A}_H(u(s+s_k,t)) = \lim_{s\rightarrow \infty} \mathcal{A}_H(u(s,t)) = \mathcal{A}_H(y) .\]
This implies that the function $s \mapsto \mathcal{A}_H(v(s,\cdot))$ is constant. Therefore, by definition, $E(v) = 0$. But, as we saw in the remark \ref{remark:floer_energy_zero}, $v$ has to be a critical point of $\mathcal{A}_H$. Therefore, $v=y$.
\end{proof}

With this lemma, we can prove the theorem \ref{theo:floer_endpoints}.

\begin{proof} {\it (Theorem \ref{theo:floer_endpoints}):} The topology of $\mathcal{L}M$ is metrizable (see \cite{hirsch2012differential} at chapter 2, section 4 for more information on this fact), so we can choose a metric $d_{\infty}$ in $\mathcal{L}M$ defining its topology. For each $x \in \crit(\mathcal{A}_H)$ we consider the open ball
\[B(x,\e) = \left\{ \gamma \in \mathcal{L}M \ | \ d_{\infty}(x,\gamma) < \e \right\} .\]
As  $\crit(\mathcal{A}_H)$ is finite, we can choose $\e > 0$ such that these balls are disjoint. In this case, let
\[U_{\e} = \bigcup_{x \in \crit(\mathcal{A}_H)} B(x,\e) \subset \mathcal{L}M .\]
Take some $u \in \mathcal{M}$. For each $\e > 0$ there exists $s_{\e}$ such that $u(s,\cdot) \in U_{\e}$ for all $s > s_{\e}$. If this were not true, then would be able to build a sequence $s_k$ with $u \cdot s_k \notin U_{\e}$ for all $k$, getting a contradiction with lemma \ref{lema:floer_sequence_limit}.

Moreover, again by lemma \ref{lema:floer_sequence_limit}, there exists some $y \in \crit(\mathcal{A}_H)$ such that $u(s,\cdot) \in B(y,\e)$ for all $s > s_{\e}$. However, this is precisely the definition of convergence in a metric space, so
\[\lim_{s \rightarrow \infty} u(s,\cdot) = y \in \crit(\mathcal{A}_H) .\]
In addition, by the proposition \ref{floereq_ellipticreg}, we know that
\[\lim_{s\rightarrow\infty} \frac{\partial u}{\partial t}(s,\cdot) = \dot{y} .\]
Therefore,
\[\lim_{s\rightarrow\infty} \frac{\partial u}{\partial s}(s,\cdot) = \lim_{k\rightarrow\infty} \left( - J \frac{\partial u}{\partial t}(s,\cdot) + JX_t(u(s,\cdot)) \right) = -J\cdot{y} + JX_t(y) =\]
\[= J(X_t(y) - \dot{y}) = 0 .\]
\end{proof}

\begin{coro} If $u \in \mathcal{M}$ and $1 \leq k \leq m$, then
\[\lim_{s\rightarrow\infty} \frac{\partial^m u}{\partial^k s \partial^{m-k} t}(s,t) = 0\]
uniformly in $t$.
\end{coro}
