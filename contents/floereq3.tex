\section{The Floer equation}

In this section we are going to introduce the analogous concept to the one of a pseudogradient of a function in Morse homology. In particular, it will be enough for us to construct the gradient of the action functional. However, this will not be as easy as the case of finite dimensions.

Take, for the compact symplectic manifold $(M,\omega)$, an almost complex structure $J$ calibrated by $\omega$, as defined in \ref{definition:calibrated_almost_complex}. This induces a Riemannian metric $g$ on $M$. We can use it to induce an inner product on each fiber of $T\mathcal{L}M$, by
\[\langle X, Y \rangle_x = \int_0^1 g_{x(t)} (X(t), Y(t)) d t,\]
for $X,Y \in T_x\mathcal{L}M$.

It is an inner product because $g_{x(t)}$ is at each $t \in \con{S}^1$, so the bilinearity, symmetry and positive definition are preserved.

Take into account that the tangent spaces are not complete, so this inner product does not induce a Hilbert space structure on each fiber $T_x\mathcal{L}M$. Nonetheless, we can define the gradient of a functional without assuming that it exists, and trying to understand under which conditions is it actually defined.

\begin{deff}
Let $F : \mathcal{L}M \rightarrow \R$ a functional. The {\bf gradient} of $F$, if it exists, is the vector field $\grad F \in \mathfrak{X}(\mathcal{L}M)$ such that, for every loop $x \in \mathcal{L}M$ and every $Y \in \mathfrak{X}(\mathcal{L}M)$,
\[\langle \grad F, Y \rangle_x = d F(x) \cdot Y ,\]
so
\[\int_0^1 g_{x(t)}(\grad_xF(t),Y(t)) = d F(x) \cdot Y .\]
\end{deff}

In the case of the action functional, its gradient (if it exists) satisfies that
\[d \mathcal{A}_H(x) \cdot Y = \langle \grad\mathcal{A}_H, Y \rangle_x = \int_0^1 g_{x(t)}(\grad_x\mathcal{A}_H(t),Y(t)) d t ,\]
and
\[d \mathcal{A}_H(x) \cdot Y = \int_0^1 \omega(\dot{x}(t)-X_H,Y(t)) d t ,\]
and, as $g(v,w) = \omega(v,Jw)$, we deduce that
\[\grad_x\mathcal{A}_H = J\dot{x} - J X_H .\]

Let $\mathcal{X}_H$ denote the negative gradient, so $\mathcal{X}_H = - \grad\mathcal{A}_H = J X_H - J \dot{x}$. Recall that, by proposition \ref{prop:symplectic_gradient}, $X_H = J \grad H$. Therefore, we get that
\[\mathcal{X}_H(x)(t) = - \grad H(x(t)) - J_{x(t)} \dot{x}(t) .\]

Let us consider $u : (-\e,\e) \rightarrow \mathcal{L}M$ an integral curve of this vector field. In other words, $u : (-\e,\e) \times \con{S}^1 \rightarrow M$ is a smooth map with
\[\frac{\partial u}{\partial s} (s,t) = \mathcal{X}_H(u(s,\cdot)) (t) .\]
From this, we are in position to define the {\bf Floer equation}:

\begin{equation} \label{equation:floer_equation}
\frac{\partial u}{\partial s} + J \frac{\partial u}{\partial t} + \grad H(u) = 0 .
\end{equation}

If we write all the variables we see that it is a non-linear partial differential equation:
\[\frac{\partial u}{\partial s}(s,t) + J_{u(s,t)} \frac{\partial u}{\partial t}(s,t) + \grad H(u(s,t)) = 0 .\]

This equation will be our main focus of study, since its solutions (if there are any) are the flow lines of the negative gradient of the action functional, in the same way that we studied the trajectories of the negative gradient flow in the Morse case.

Let us begin with some remarks on the equation. In particular, its “degenerate” cases:

\begin{rmrk}
If $H$ is a constant function (so $\grad H = 0$), then the equations are the Cauchy-Riemann equations (for almost complex structures):
\[\frac{\partial u}{\partial s} + J \frac{\partial u}{\partial t} = 0\]
\end{rmrk}

\begin{rmrk} \label{rmrk:floer_stationary}
The only stationary solutions (such that $\partial_s u = 0$) of the equation are the 1-periodic orbits of $X_H$:
\[0 = J \frac{\partial u}{\partial t} + \grad H \Rightarrow \frac{\partial u}{\partial t} = X_H ,\]
where we used proposition \ref{prop:symplectic_gradient} again.
\end{rmrk}

\begin{rmrk}
If we assume that a solution does not depend on $t$ (this means, $u(s,t) = u(s)$, so all the points are degenerate loops, this means, points in $M$), then we find ourselves in the case of Morse theory of critical points again, because Floer equation reads
\[\frac{\partial u}{\partial s} = - \grad H .\]
\end{rmrk}

Recall that the negative gradient flow lines in Morse theory are defined in order to connect critical points. However, in the case of the Floer equation, this may not be the case. This means, if $u$ is a solution to the equation, it is not necessarily true that $\displaystyle\lim_{s \rightarrow -\infty} u(s,\cdot)$ or $\displaystyle\lim_{s \rightarrow \infty} u(s,\cdot)$ exist or are critical points of $\mathcal{A}_H$.

However, we just need to add a condition to our equation in order to have this property. In order to justify it, recall that when we proved proposition \ref{prop:connect_critical_points} (which is the analogous result in Morse theory) we used the energy of a trajectory $\gamma$ in order to prove our result:
\[E(\gamma) = - \int_{\R} \gamma^{\ast} d f = \int_{+\infty}^{-\infty} \frac{d}{d t} (f \circ \gamma) d t .\]

This can be defined similarly in the case of the solutions of the Floer equation that are defined for all $s \in \R$:

\begin{deff} \label{definition:floer_energy}
Let $u : \R \times \con{S}^1 \rightarrow M$ a solution of the Floer equation. Its energy (which may be infinite) is the integral
\[E(u) = - \int_{\R} u^{\ast} d \mathcal{A}_H = - \int_{\R} \frac{d}{d s} \left( \mathcal{A}_H \circ u \right) d s .\]
\end{deff}

If we expand the last term, we get that
\[E(u) = - \int_{\R} d \mathcal{A}_H(u(s)) \cdot \frac{\partial u}{\partial s} d s = - \int_{\R} g \left(\grad\mathcal{A}_H(u(s)),\frac{\partial u}{\partial s} \right) d s = \int_{\R \times \con{S}^1} \left| \frac{\partial u}{\partial s} \right|^2 d t d s .\]
Therefore, the space we are interested in is the one of solutions with finite energy:

\begin{deff} \label{definition:finite_energy_space}
The {\bf space of solutions with finite energy} is
\[\mathcal{M} = \left\{ u : \R \rightarrow \mathcal{L}M \ | \ u \text{ solves \ref{equation:floer_equation} and } E(u) < \infty \right\} .\]
\end{deff}

\begin{rmrk} \label{remark:floer_energy_zero}
If $u \in \mathcal{M}$ and $E(u) = 0$, then $u$ does not depend on $s$. By the remark \ref{rmrk:floer_stationary}, in this case $u$ is a loop, and a critical point of $\mathcal{A}_H$.
\end{rmrk}

\begin{rmrk} \label{remark:floer_limits}
If $u \in \mathcal{M}$ and there are loops $x,y \in \mathcal{L}M$ with
\[\lim_{s \rightarrow -\infty} u(s) = y, \ \ \lim_{s \rightarrow \infty} u(s) = x ,\]
then we have that
\[E(u) = \mathcal{A}_H(y) - \mathcal{A}_H(x) ,\]
just as in the Morse homology case.
In Section \ref{section:floer5} we will see that this is always the case, this means, we always have that $u$ has limiting loops $x$ and $y$.
\end{rmrk}
