In Chapter 1 we presented the Morse homology, stressing out how we obtain a connection between Morse functions and the topology of the manifold. In particular, in Section \ref{section:morse_complex} we saw a layout of how to construct this kind of homology. We have tried to keep that layout in mind all the time while doing Morse theory, because in the context of Floer homology we reproduced analogous steps. Then, we moved on to a review on symplectic geometry and the Arnold conjecture. A special case of the Arnold conjecture, when the Hamiltonian does not depend on time (this means, in an autonomous system), came as a corollary of the Morse theory.

To tackle the general case we defined, in Chapter 3, the basic ingredients of Floer homology: the action functional, the Floer equation, the energy of a solution of the Floer equation, and the space $\mathcal{M}$ of solutions of finite energy. Recall that the Floer equation is a perturbed version of the Cauchy-Riemann equation, and so when we studied the space $\mathcal{M}$...

The first main result derived from the Floer homology is the one it was intended for in the first place:

\begin{rmrk}
Let $M$ be a compact, symplectic manifold. Suppose that it is aspherical, this means, that it satisfies the assumptions \ref{assumption1} and \ref{assumption2}. Then, the conjecture \ref{theo:arnold_conjecture} is true: for any time-dependent Hamiltonian $H_t$, the number of non-degenerate periodic orbits of the flow $X_H$ is greater or equal than
\[\sum_k \mathrm{dim}(H_k(M)) .\]
\end{rmrk}

Nonetheless, Floer homology is a versatile tool that plays an important role in modern symplectic topology. Its implications are still being studied.

%TODO: Dir que hem fet.
