In this chapter we are going to provide some proofs that were not included in Chapter 1 because they were rather too technical, and including them would not help to our principal goal to focus on the scheme of the proof rather than going into the details of the theory. However, these results are important enough to be written in this appendix

\section{The Morse functions are generic}

In section \ref{section:morse_functions} we asked ourselves if any manifold $M$ admits Morse functions, and if this functions are generic enough. In this section we will provide an affirmative answer to both questions. To prove the existence, we are going to need two fundamental theorems of differential geometry:

\begin{theo}
{\bf (Whitney embedding theorem):} Any smooth manifold of dimension $n$ can be smoothly embedded in $\R^{2n}$, if $n > 0$.
\end{theo}

\begin{theo}
{\bf (Sard's theorem):} Let $f : M \rightarrow N$ a smooth map. Then, the set of critical values of $f$ has measure zero.
\end{theo}

Therefore, we can think of any manifold $M$ as a smooth submanifold of $\R^N$, for some $N$. This allows us to state the following proposition:

\begin{prop} \label{existenceMorse}
Let $M \subset \R^N$ a submanifold. For almost every point $p \in \R^N$, the function

\begin{displaymath}
	\begin{array}{rccc}f_p : & M & \longrightarrow & \R \\ & x & \longmapsto & \|x-p\|^2 \end{array}
\end{displaymath}

is a Morse function.
\end{prop}

\begin{proof} Let $(u_1,...,u_d) \mapsto x(u_1,...,u_d)$ a local parametrization from $\R^d$ into $M \subset \R^N$ in a neighbourhood of a point $p \in M$. In this coordinates, the partial derivatives of the function $f_p$ are

\begin{displaymath}
\frac{\partial f_p}{\partial u_i} = 2(x-p) \cdot \frac{\partial x}{\partial u_i} ,
\end{displaymath}

and

\begin{displaymath}
\frac{\partial^2 f_p}{\partial u_i \partial u_j} = 2 \left(\frac{\partial x}{\partial u_i} \cdot \frac{\partial x}{\partial u_j} + (x-p) \cdot \frac{\partial^2 x}{\partial u_i \partial u_j} \right) .
\end{displaymath}

The point $x$ is therefore a non-degenerate critical point if and only if $(x-p)$ is orthogonal to $\straightT_xM$ and the matrix $\frac{\partial^2 f_p}{\partial u_i \partial u_j}$ has rank $d$.

To show that $f_p$ is a Morse function for almost all $p \in \R^N$ it suffices to show that that the $p$ that do not satisfy the condition are the critical values of a smooth map, and then apply the Sard's theorem. Consider the normal fiber bundle of the embedding of $M$ into $\R^N$, which is a smooth manifold:

\begin{displaymath}
N = \{(x,v) \in M \times \R^N \ | \ v \bot \straightT_xM \} ,
\end{displaymath}

and the map

\begin{displaymath}
\begin{array}{rccc} E : & N & \longrightarrow & \R^N \\ & (x,v) & \longmapsto & x+v \end{array} .
\end{displaymath}

Then, it sufices to proof the following lemma:

\begin{lema}
The point $p = x + v$ is a critical value of $E$ if, and only if, the matrix

\begin{displaymath}
\frac{\partial^2 f_p}{\partial u_i \partial u_j} = 2 \left( \frac{\partial x}{\partial u_i} \cdot \frac{\partial x}{\partial u_j} - v \cdot \frac{\partial^2 x}{\partial u_i \partial u_j} \right)
\end{displaymath}

is not invertible.
\end{lema}

\begin{proof}
Consider, for each point of the local chart $(u_1,...,u_d)$, a orthonormal basis of $(\straightT_xM)^{\bot}$ by the $N-d$ vectors $v_1,...,v_{N-d}$. Then, we have a local parametrization for $N$, given by the map

\begin{displaymath}
(u_1,...,u_d,t_1,...,t_{N-d}) \longmapsto \left( x(u_1,...,u_d), \sum_{i=1}^{N-d} t_i v_i (u_1,...,u_d) \right) .
\end{displaymath}

In these coordinates, the partial derivatives of $E$ are

\begin{displaymath}
\left\{ \begin{array}{l} \frac{\partial E}{\partial u_i} = \frac{\partial x}{\partial u_i} + \sum_{k=1}^{N-d} t_k \frac{\partial v_k}{\partial u_i} \\ \frac{\partial E}{\partial t_j} = v_j \end{array} \right. .
\end{displaymath}

If we compute the inner products of these $N$ vectors with the $N$ independent vectors $\frac{\partial x}{\partial u_1} , ..., \frac{\partial x}{\partial u_d}, v_1, ..., v_{N-d}$, we get a square matrix that has the same rank as the Jacobian of $E$, and this matrix has the form

\begin{displaymath}
\begin{pmatrix} \left( \displaystyle\frac{\partial x}{\partial u_i} \cdot \frac{\partial x}{\partial u_j} + \sum_k t_k \frac{\partial v_k}{\partial u_i} \cdot \frac{\partial x}{\partial u_j} \right) & \left( \displaystyle\sum_k \frac{\partial v_k}{\partial u_i} \cdot v_l \right) \\ 0 & \text{Id} \end{pmatrix} .
\end{displaymath}

But $v_k$ are orthogonal to $\frac{\partial x}{\partial u_j}$, so

\begin{displaymath}
0 = \frac{\partial}{\partial u_i} \left( v_k \cdot \frac{\partial x}{\partial u_j} \right) = \der{v_k}{u_i} \cdot \der{x}{u_j} + v_k \cdot \frac{\partial^2 x}{\partial u_i \partial u_j} ,
\end{displaymath}

so

\begin{displaymath}
\sum_k t_k \der{v_k}{u_i} \cdot \der{x}{u_j} = - \sum_k t_k v_k \frac{\partial^2 x}{\partial u_i \partial u_j} = v \cdot \frac{\partial^2 x}{\partial u_i \partial u_j} .
\end{displaymath}

And, thus, the lemma is proved.
\end{proof}

So the proposition is proved.
\end{proof}

To answer the second question, about the genericness of Morse functions on a given manifold, we have the following proposition:

\begin{prop} \label{proposition:morse_generic}
Let $M$ be a smooth manifold, and let $f : M \rightarrow \R$ a smooth function. Let $k$ be a positive integer. Then, $f$ and all its derivatives of order $\leq k$ can be uniformly approximated by Morse functions on every compact subset.
\end{prop}

\begin{proof}
Choose an embedding of $M$ into $\R^N$ such that its first coordinate is the function $f$,

\begin{displaymath}
h(x) = (f(x),h_2(x),...,h_N(x)) .
\end{displaymath}

Choose $c$ a (large) real number. By the proposition \ref{existenceMorse}, for almost all point $p = (-c+\e_1,\e_2,...,\e_N)$ the function $f_p$ is a Morse function, so the function

\[g_{c,\e}(x) = \frac{f_p(x)-c^2}{2c}\]

is also a Morse function. Moreover, notice that

\[g_{c,\e}(x) = \frac1{2c} \left((f(x) + c - \e_1)^2 + (h_2(x)-\e_2)^2 + \cdots + (h_N(x)-\e_N)^2 - c^2 \right) =\]

\[= f(x) + \frac{f(x)^2 + \sum h_i(x)^2}{2c} - \frac{\e_1f(x) + \sum \e_ih_i(x)}{c} + \sum \e_i^2 - \e_1 ,\]

so, as claimed, on any compact subset of $M$, $g_{c,\e}$ tends to $f$ and its first $k$ derivatives also tend to the first $k$ derivatives of $f$ as $\e \rightarrow 0$ and $c \rightarrow \infty$.
\end{proof}

Therefore, the Morse functions are actually generic, in the sense that we can approximate any smooth function with Morse functions over compact subsets of $M$.

\section{The broken trajectories are the boundary in dimension 1} \label{appendix:brokenboundary}

In this section, we are going to focus on the proof of theorem \ref{morse_brokenboundary}. This theorem is fundamental to the construction of the Morse complex, because it is, ultimately, the proof that it is indeed a complex, this means, that $\partial_X^2 = 0$. Let us restate it here, with more precision. Recall that we have a triple $(M, f, X)$, where $M$ is a compact smooth manifold, $f$ is a Morse function defined on $M$, and $X$ is a pseudogradient adapted to $f$ and satisfying the Smale condition.

\begin{theo}
Let $a \in \crit_{k+1}(f), b \in \crit_k(f), c \in \crit_{k-1}(f)$. For all $(\lambda_1,\lambda_2) \in \mathcal{L}(a,b) \times \mathcal{L}(b,c)$, $\exists \psi : [0,\delta) \rightarrow \overline{\mathcal{L}}(a,c)$ (for some $\delta > 0$) such that

\begin{enumerate}
	\item $\psi(0) = (\lambda_1,\lambda_2)$.
	\item $\psi(t) \in \mathcal{L}(a,c) \ \forall t > 0$.
	\item $\left. \psi \right|_{\{t > 0\}} : (0,\delta) \rightarrow \mathcal{L}(a,c)$ is an embedding.
	\item For all $(l_n)_n \subset \mathcal{L}(a,c)$ with $l_n \xrightarrow[n \rightarrow \infty]{} (\lambda_1, \lambda_2)$ in $\overline{\mathcal{L}}(a,c)$, $l_n \in \psi((0,\delta)) \ \forall n$ (at least, for $n$ large enough).
\end{enumerate}

This means that $\overline{\mathcal{L}}(a,c)$ is a finite union of compact and connected smooth manifolds with boundary.
\end{theo}

\begin{proof}
Take $\alpha = f(b)$. Let $\Omega(b) \subset M$ be a Morse chart for $(f,X)$, such that $\partial\Omega(b)$ coincides with $f^{-1}(\alpha+\e)$ in a neighbourhood of $W^s(b) \cap \partial\Omega(b)$ and with $f^{-1}(\alpha-\e)$ in a neighbourhood of $W^u(b) \cap \partial\Omega(b)$ for some $\e > 0$.

Let $b^- = \partial\Omega(b) \cap \lambda_1$ be the entry point of $\lambda_1$ into $\Omega(b)$, and $b^+ = \partial\Omega(b) \cap \lambda_2$ be the exit point of $\lambda_2$. Take $U \subset \partial\Omega(b)$ a neighbourhood of $b^-$. Notice that $U$ is diffeomorphic to an open disk of dimension $n-1$. Let us define the following sets, all of them in $\partial\Omega(b)$:

\begin{itemize}
	\item $P := U \cap W^u(a)$. The unstable manifold meets $f^{-1}(\alpha+\e)$ transversally, so $P$ is diffeomorphic to an open disk of dimension $\text{dim}(P) = \text{dim}(U) + \text{dim}(W^u(a)) - \text{dim}(M) = k$.
	\item $S_+(b) = U \cap W^s(b)$. For the same reason, $\text{dim}(S_+(b)) = n - k - 1$, and it is diffeomorphic to a sphere.
	\item $S_-(b) = W^u(b) \cap f^{-1}(\alpha-\e)$. It is diffeomorphic to a sphere of dimension $k-1$.
\end{itemize}

As $X$ satisfies the Smale condition, we know that $W^u(a) \pitchfork W^s(b)$. Therefore, $P \pitchfork S_+(b)$ in $U$, so $\text{dim}(P \cap S_+(b)) = 0$. Therefore, (shrinking $U$ if necessary) we can assume that $P \cap S_+(b) = \{b^-\}$.

Take $D = P \backslash \{b^-\}$. It is diffeomorphic to $\{x \in \R^{k-1} \ | \ 0 < \|x\| < 1\}$, a punctured open disk. Notice that, by definition, $D \cap W^s(b) = \emptyset$. Therefore, the flux of $X$ starting at any point of $D$ will eventually leave $\Omega(b)$, so we can define the embedding

\[\begin{array}{rccc} \Phi : & D & \longrightarrow & \partial\Omega(b) \end{array}\]

induced by this flux.

Let us consider the set $Q = \text{Im} \Phi \cup S_-(b) \subset f^{-1}(\alpha-\e)$. The key to prove this theorem is contained in the following proposition:

\begin{prop} \label{morse_propappendix}
Q is a $k$-dimensional manifold with boundary, and $\partial Q = S_-(b)$.
\end{prop}

Let us suppose that it is true. Using the Smale transversality condition of $X$ in $W^u(a) \cap W^s(c)$ and in $W^u(b) \cap W^s(c)$, we can see that, as $\text{Im}\Phi \subset W^u(a) \cap f^{-1}(\alpha-\e)$ and $S_-(b) = W^u(b) \cap f^{-1}(\alpha-\e)$,

\[\text{dim}(\text{Im}\Phi \cap W^s(c)) = 1 ,\]

\[\text{dim}(S_-(b) \cap W^s(c)) = 0 .\]

Therefore, $Q \cap W^s(c)$ is a 1-dimensional manifold with boundary, and its boundary is, by the proposition \ref{morse_propappendix},

\[\partial Q \cap W^s(c) = S_-(b) \cap W^s(c) = W^u(b) \cap W^s(c) \cap f^{-1}(\alpha-\e) \cong \mathcal{L}(b,c) ,\]

and $b^+ \in \partial Q \cap W^s(c)$. Now, consider $\chi$ a local parametrization of this manifold in a neighbourhood of $b^+$, this means, an embedding

\[\begin{array}{rccc} \chi : & [0,\delta) & \longrightarrow & Q \cap W^s(c) \end{array} ,\]

with $\chi(0) = b^+$. Then, we can consider the diffeomorphism

\[\begin{array}{rccc} \Phi^{-1} \circ \chi : & (0,\delta) & \longrightarrow & W^s(c) \cap D \end{array}\]

(it is well defined because $\Phi$ is defined following the flow lines of $X$, so $W^s(c)$ is invariant under its action). We can use the lemma \ref{morse_sequences} to deduce that

\[\lim_{t \searrow 0} (\Phi^{-1} \circ \chi)(t) = b^- ,\]

so we can extend $\Phi^{-1} \circ \chi$ continuously to a map

\[\begin{array}{rccc} \psi : & [0,\delta) & \longrightarrow & \left( W^s(c) \cap P \right) \cup \{b^-\} \end{array}\]

with $\psi(0) = b^-$. We can see that

\[W^s(c) \cap P = W^s(c) \cap W^u(a) \cap f^{-1}(\alpha+\e) \cong \mathcal{L}(a,c) ,\]

and $b^-$ is a natural representation of $(\lambda_1,\lambda_2) \in \mathcal{L}(a,b) \times \mathcal{L}(b,c)$, so we can rewrite the definition of $\psi$ as

\[\begin{array}{rccc} \psi : & [0,\delta) & \longrightarrow & \overline{\mathcal{L}}(a,c) \end{array} .\]

Moreover, we see that:

\begin{itemize}
	\item $\psi(0) = (\lambda_1,\lambda_2)$.
	\item $\psi(t) \in \mathcal{L}(a,c)$ for $t > 0$.
	\item $\left. \psi \right|_{\{t > 0\}}$ is an embedding.
\end{itemize}

We just need to prove the property 4 to conclude the proof of the theorem.

Consider $l_n \xrightarrow[n \rightarrow \infty]{} (\lambda_1,\lambda_2)$, with $l_n \in \mathcal{L}(a,c) \ \forall n$. For $n$ sufficiently large, $l_n$ enters $\Omega(b)$ through $U$, and exits it through a neighbourhood of $b^+$ in $f^{-1}(\alpha-\e)$. Let $l_n^-$ denote the entry points, and $l_n^+$ the exit points. By the lemma \ref{morse_sequences}, $l_n^- \xrightarrow[n \rightarrow \infty]{} b^-$, and $l_n^+ \xrightarrow[n \rightarrow \infty]{} b^+$.

For $n$ large enough, $l_n^- \in D$, so it is in the domain of $\Phi$. Thus, $l_n^+ = \Phi(l_n^-) \in Q \cap W^s(c)$, and therefore $l_n^+ \in \text{Im} \chi$. Then, it is clear that $l_n \in \text{Im} \psi$, as we wanted to see.
\end{proof}

Now let us prove the proposition \ref{morse_propappendix}. We want to conclude that $Q$ is a $k$-dimensional manifold with boundary, and its boundary is precisely $S_-(b)$.

\begin{proof}
{\it (Proposition \ref{morse_propappendix}):} Recall that $Q = \text{Im}\Phi \cup S_-(b)$, with $\Phi : D \rightarrow f^{-1}(\alpha-\e)$ an embedding. Also, $D = P \backslash \{b^-\}$ is diffeomorphic to a punctured open disk of dimension $k-1$. This means that we can take $D \cong (0,1) \times \con{S}^{k-1}$. Therefore, we can take coordinates $(t,z)$ in $D$, with $t \in (0,1)$ and $z \in \con{S}^{k-1}$. We can construct this coordinates in a way that $\displaystyle\lim_{t \searrow 0} (t,z) = b^-$ in $P$ for all $z \in \con{S}^{k-1}$.

We want to show that, on the other hand,
\[\displaystyle\lim_{t \searrow 0} \Phi((t,z)) \in S_-(b) \ \forall z \in \con{S}^{k-1}. \]

Let $d^+ = \displaystyle\lim_{s \searrow 0} (s,w)$ for some $w \in \con{S}^{k-1}$. It is clear that $f(d^+) = \alpha - \e$ because $f$ is continuous, so we need to show that $d^+ \in W^u(b)$.

Suppose that it is false, so $d^+ \notin W^u(b)$. Therefore, if we take the flow of $- X$ starting at $d^+$, we must exit the chart $\Omega(b)$ at some point $d^-$ with $f(d^-) = \alpha + \e$. Let $(x_n)_n$ a sequence in $\text{Im}\Phi$ such that $\limn x_n = d^+$. Take $y_n = \Phi^{-1}(x_n)$ for each $n$. We know that $f(y_n) = \alpha + \e \ \forall n$, so, applying the lemma \ref{morse_sequences}, we deduce that $y_n \xrightarrow[n \rightarrow \infty]{} d^-$. However, we can see that $y_n \xrightarrow[n \rightarrow \infty]{} b^-$ because the first component tends to $0$. Therefore, we conclude that $b^- = d^-$. But we can see that, if $\varphi_X^t$ denotes the flow of $X$,

\[f(\varphi_X^t(b^-)) \xrightarrow[t \rightarrow \infty]{} \alpha ,\]

so $f(\varphi_X^t(b^-)) \geq \alpha \ \forall t > 0$. On the other hand, $f(d^+) = \alpha - \e < \alpha$ and $d^+ = \varphi_X^{t_0}(d^-)$ for some $t_0 > 0$. Thus, we reach a contradiction.

We know that $S_-(b) \cong \con{S}^{k-1}$, so we can identify $Q$ with $[0,1) \times \con{S}^{k-1}$ and $S_-(b)$ with $\{0\} \times \con{S}^{k-1}$ with the same coordinates as before. We just showed that we can define the map
\[\begin{array}{rccc}\rho : & \con{S}^{k-1} & \longrightarrow & S_-(b) \\ & z & \longmapsto & \displaystyle\lim_{t \searrow 0} \Phi((t,z)) \end{array}\]

and it is indeed well defined. If we show that $\rho$ is bijective, we will be able to conclude that $S_-(b)$ is precisely the boundary of the manifold $Q$, and we will be done. In fact, as $S_-(b) \cong \con{S}^{k-1}$, it sufices to show that $\rho$ is injective.

To do this, we need to use the fact that $\Omega(b)$ is a Morse chart and, by the Morse Lemma \ref{morselemma}, it admits a coordinate system $(x_-,x_+)$\footnote{This means, with $(x_-,x_+)=(x_-^1,...,x_-^k,x_+^1,...,x_+^{n-k})$.} such that $f(x_-,x_+) = c - \|x_-\|^2 + \|x_+\|^2$. Then, it can be easily proved that the flow of $X$ (which coincides with that of $-\grad f$ in $\Omega(b)$ because of the definition of a pseudogradient adapted to $f$) takes the form
\[\varphi_X^s(x_-,x_+) = (e^{2s}x_-,e^{-2s}x_+) .\]

On the other hand, we can see that, for some $\delta > 0$,
\[Q = \{(x_-,x_+) \ | \ \|x_-\|^2 = \e, 0 \leq \|x_+\| < \delta\}, \]

and that
\[D = \{(x_-,x_+) \ | \ \|x_+\|^2 = \e, 0 < \|x_-\|^2 < \delta' \}\]

for some $\delta'$ that depends on $\delta$, as we are going to show. With all of this, it is possible to compute the $s$ such that $\varphi_X^s(x) \in \text{Im}\Phi$ for a given $x \in D$, so we can deduce the form of $\Phi$ in these coordinates:
\[\Phi((x_-,x_+)) = \left(\frac{\|x_+\|}{\|x_-\|} x_-, \frac{\|x_-\|}{\|x_+\|} x_+\right) .\]

This way, if we let $(x_-,x_+) \in D$, we have that $(x_-,x_+) = ((t,z), \sqrt{\e})$ (where $\sqrt{\e}$ denotes the vector with all the coordinates equal to $\sqrt{\e}$, and $(t,z)$ are the polar coordinates that we introduced before). Then, we conclude that
\[\Phi((t,z)) = (\sqrt{\e} z, t) .\]

Therefore, $\rho(z) = \displaystyle\lim_{t \searrow 0} \Phi((t,z)) = (\sqrt{\e} z, 0)$, so the map $\rho$ is injective, as we wanted to prove.
\end{proof}
