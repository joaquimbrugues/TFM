\section{Morse functions and pseudogradients}

\subsection{Morse functions}

Let us consider a differentiable manifold $M$ without boundary, and a $\mathcal{C}^{\infty}$ - function to $\R$ defined on this manifold. Recall that

\begin{deff}
	A point $p \in M$ is called a {\bf critical point} of the function $f$ if the tangent map $\dd f_p : \straightT_pM \longrightarrow \straightT\R \cong \R$ is zero. In this case, we sat the $f(p)$ is a {\bf critical value}.
\end{deff}

To classify the critical points of a function, we will be interested in the directions in which it is convex and the ones in which it is concave. The best way to study the convexity of a function systematically is its Hessian, or second derivative. However, we find ourselves with a constraint: it is not possible to define the Hessian of a function in all points in a way that it is independent of the choice of the coordinate system! Fortunatelly, it is possible to do it at each critical point, as we will show.

\begin{deff}
	Let $p$ be a critical point of a function $f \in \mathcal{C}^{\infty}(M)$. The {\bf Hessian} of $f$ at $p$ is the bilinear map

\begin{displaymath}
	\begin{array}{rccc} \straightH_p[f] : & \straightT_pM \times \straightT_pM & \longrightarrow & \R \\ & (u,v) & \longmapsto & v(X_u(f)) \end{array} ,
\end{displaymath}

where $X_u$ is a vector field extending $u \in \straightT_pM$ locally.
\end{deff}

To define the extension of a vector to a vector field, we use this lemma

\begin{lema}
	Let $u \in \straightT_pM$. There is a vector field $X_u \in \mathfrak{X}(M)$ such that $X_u(p) = u$.
\end{lema}

This lemma can be proved in a straightforward way using a suitable bump function defined inside a local chart of $p$ in $M$.

\begin{lema}
	The Hessian $\straightH_p[f]$ is well defined (this means, it does not depend on the choice of $X_u$), and it is a bilinear and symmetric map.
\end{lema}

\begin{proof}
	First, we show that it is symmetric, from where it will be clear that it is well defined:

$$\straightH_p[f](v,w) - \straightH_p[f](w,v) = v(X_w(f)) - w(X_v(f)) = \left. X_v \right|_p(X_w(f)) - \left. X_w \right|_p (X_v(f)) = $$ 
$$= \left. [X_v,X_w] \right|_p(f) = \dd_p f \cdot [X_v,X_w](p) = 0 ,$$

where the last term is zero because $\dd_p f = 0$, as $p$ is a critical point for $f$. Thus, if we choose extensions $X_v$ and $X_w$ for $v$ and $w$ (respectivelly), then $\straightH_p[f](v,w) = \straightH_p[f](w,v)$.

Looking again at the intrinsic definition,

$$\straightH_p[f](v,w) = v(X_w(f)),$$

it is obvious that this does not depend on the extension $X_v$ that we choose for $v$, as in the expression only depends on $X_v(p) = v$ regardless of the extension. On the other hand, as we just proved, $\straightH_p[f](v,w) = \straightH_p[f](w,v)$, so, applying the same argument, the Hessian does not depend on the extension chosen for $w$. This proves that the Hessian is well defined.

\

Finally, the Hessian is bilinear, because
$$H_p[f](\alpha u + \beta v, w) = (\alpha u + \beta v)(X_w(f)) = $$
$$= \alpha u(X_w(f)) + \beta v(X_w(f)) = \alpha H_p[f](u,w) + \beta H_p[f](v,w) ,$$

and the same argument applies to the second component by symmetry.
\end{proof}

Notice that the last proof depends on the fact that $p$ is a critical point. This means, in general it is not possible to proof that $\straightH_p[f]$ is well defined, when $p$ is not a critical point.

\begin{rmrk}
	The local form of the Hessian of a function coincides with the Hessian of the local representation of the function in a chart. If $(x_1,...,x_n)$ is a local chart centered in a point $p \in M$ and $\tilde{f}$ is the local representation of $f$ in this chart, then the local expression of $\straightH_p[f]$ is preciselly the matrix

\begin{displaymath}
	\tilde{\straightH}_p[f] := \left( \frac{\partial^2 \tilde{f}}{\partial x_i \partial x_j}(p) \right)_{i,j} .
\end{displaymath}
\end{rmrk}

As we said before, we are interested in the Hessian of a function at a critical point to, in some way, count the number of directions in which the function is convex. To do so, we need the concepts of index and non-degeneracy.

\begin{deff}
	We define the

\begin{itemize}
	\item {\bf Index} of $p$ as the dimension of the maximal subspace $V \subset \straightT_pM$ such that $\left. \straightH_p[f] \right|_V$ is negative definite.
	\item {\bf Nullity} of $p$ as the dimension of the null-space of $\straightH_p[f]$, this means, the maximal subspace $N \subset \straightT_pM$ such that $\straightH_p[f](N,\cdot) = 0$.
	\item {Non-degenerate critical points} of $f$ as the points $p$ that have nullity $0$, this means, that the local representation of $\straightH_p[f]$ has maximal rank in any local chart representation.
\end{itemize}
\end{deff}

Notice that all the definitions that we just gave are well defined, because the index and nullity of a matrix are independent of the basis chosen to represent the matrix, so they are also independent of any change of coordinates.

Therefore, it makes sense to classify the critical points of a manifold according to their index.

\begin{deff}
	We say that a function $f \in \mathcal{C}^{\infty}(M)$ that has only non-degenerate critical points is a {\bf Morse function}.
	If $f$ is a Morse function, we denote

	\begin{displaymath}
		\crit_k(f) = \{p \in M \ | \text{ $p$ has index $k$}\} ,
	\end{displaymath}
	\begin{displaymath}
		\crit(f) = \bigcup_{k \geq 0} \crit_k(f) .
	\end{displaymath}
\end{deff}

Now that we have presented the Morse functions, we are ready to study their behaviour in a neighbourhood of a critical point:

\begin{prop}
	{\bf (Morse Lemma):} Let $p \in \crit(f)$. Then there is a local coordinate system $(U,(y_1,...,y_n))$ centered on $p$ (this means, with $y_i(p) = 0 \ \forall i$) such that, if $k$ is the index of $p$,

$$\left. f \right|_U = f(p) - y_1^2 - ... - y_k^2 + y_{k+1}^2 + ... + y_n^2 .$$
\end{prop}

\begin{proof} First, notice that, using the fundamental theorem of calculus, a local expression $\tilde{f}$ of $f$ can be written as

$$\tilde{f}(x) = \tilde{f}(0) + \int_0^1 \frac{df(tx_1,...,tx_n)}{dt} dt = f(p) + \int_0^1 \sum_{i=1}^n x_i \der{\tilde{f}}{x_i}(tx_1,...,tx_n) dt .$$

If we take $g_i(x) = \int_0^1 \der{\tilde{f}}{x_i}(tx_1,...,tx_n) dt$, we can write $\tilde{f}$ as

$$\tilde{f}(x) = f(p) + \sum_{i=1}^n x_i g_i(x) .$$

As $g_i(0) = \der{\tilde{f}}{x_i}(0) = 0$, we can apply the same process for each $i$, so there are functions $h_{ij}$ such that

$$g_i(x) = \sum_{j=1}^n x_j h_{ij}(x) ,$$
$$f(x) = f(p) + \sum_{i,j=0}^n x_i x_j h_{ij}(x) .$$

This functions are defined as
$$h_{ij}(0) = \frac12 \frac{\partial^2 \tilde{f}}{\partial x_i \partial x_j}(0) ,$$

so $h_{ij} = h_{ji}$.

Then, we can apply inductivelly a change of coordinates, perhaps shrinking the domain of the chart at each transformation. We describe the idea for each step:

We are in the following situation: there is a local coordinate system $(U_1, (u_1,...,u_n))$ (with $U_1 \subset U$) such that

$$f = f(p) \pm u_1^2 \pm ... \pm u_{r-1}^2 + \sum_{i,j \geq r} u_i u_i H_{ij}(u) ,$$

where $(H_{ij})_{i,j}$ form a symmetric matrix and $(H_{ij}(0))_{i,j}$ form a non-degenerate matrix. Therefore, we can apply a linear change of coordinates so that $H_{rr}(0) \neq 0$. Take $S(u) = \sqrt{|H_{rr}(u)|}$, which will be a non-vanishing positive function of $u$ in a neighbourhood $U_2 \subset U_1$ of $0$. Thus, introduce the new local coordinates $(v_1,...,v_n)$ on $U_2$ as

$$v_i = u_i \ \text{ for } i \neq r,$$
$$v_r(u) = S(u) \left[ u_r + \sum_{i > r} u_i \frac{H_{ir}(u)}{H_{rr}(u)} \right] .$$

Using the inverse function theorem we conclude that $(v_1,...,v_r)$ form an invertible and smooth set of coordinates on a neighbourhood of the origin, $U_3 \subset U_2$. Also, it can be seen that

$$f(v) = f(p) + \sum_{i \leq r} \pm v_i^2 + \sum_{i,j > r} v_i v_j G_{ij}(v) ,$$

where $G_{ij}$ are symmetric and form a non-degenerate matrix at $v = 0$.

This, by induction, concludes the proof.
\end{proof}

\begin{coro}
THe non-degenerate critical points of  a differentiable function are isolated.
\end{coro}

\begin{rmrk}
This last corollary is, of course, false for degenerate critical points. Think, for instance, of the function of $\R^2$ defined by $f(x,y) = x^2+y^2+2xy$ in the canonical coordinates. This function has a line of critical points, the one defined by $x + y = 0$, which has no isolated points.
\end{rmrk}

Having studied some nice properties about the Morse functions, a natural question that arises is if there exists a Morse function for any manifold $M$ that we want to think about, and, if it exists, in some sense how many of such functions are there.

To tackle the first question, which has an affirmative answer, we need two fundamental results of differential geometry:

\begin{theo}
{\bf (Whitney embedding theorem):} Any smooth manifold of dimension $n$ can be smoothly embedded in $\R^{2n}$, if $n > 0$.
\end{theo}

\begin{theo}
{\bf (Sard's theorem):} Let $f : M \rightarrow N$ a smooth map. Then, the set of critical values of $f$ has measure zero.
\end{theo}

Therefore, we can think of any manifold $M$ as a smooth submanifold of $\R^N$, for some $N$. This allows us to state the following proposition:

\begin{prop} \label{existenceMorse}
Let $M \subset \R^N$ a submanifold. For almost every point $p \in \R^N$, the function

\begin{displaymath}
	\begin{array}{rccc}f_p : & M & \longrightarrow & \R \\ & x & \longmapsto & \|x-p\|^2 \end{array}
\end{displaymath}

is a Morse function.
\end{prop}

\begin{proof} Let $(u_1,...,u_d) \mapsto x(u_1,...,u_d)$ a local parametrization from $\R^d$ into $M \subset \R^N$ in a neighbourhood of a point $p \in M$. In this coordinates, the partial derivatives of the function $f_p$ are

\begin{displaymath}
\frac{\partial f_p}{\partial u_i} = 2(x-p) \cdot \frac{\partial x}{\partial u_i} ,
\end{displaymath}

and

\begin{displaymath}
\frac{\partial^2 f_p}{\partial u_i \partial u_j} = 2 \left(\frac{\partial x}{\partial u_i} \cdot \frac{\partial x}{\partial u_j} + (x-p) \cdot \frac{\partial^2 x}{\partial u_i \partial u_j} \right) .
\end{displaymath}

The point $x$ is therefore a non-degenerate critical point if and only if $(x-p)$ is orthogonal to $\straightT_xM$ and the matrix $\frac{\partial^2 f_p}{\partial u_i \partial u_j}$ has rank $d$.

To show that $f_p$ is a Morse function for almost all $p \in \R^N$ it suffices to show that that the $p$ that do not satisfy the condition are the critical values of a smooth map, and then apply the Sard's theorem. Consider the normal fiber bundle of the embedding of $M$ into $\R^N$, which is a smooth manifold:

\begin{displaymath}
N = \{(x,v) \in M \times \R^N \ | \ v \bot \straightT_xM \} ,
\end{displaymath}

and the map

\begin{displaymath}
\begin{array}{rccc} E : & N & longrightarrow & \R^N \\ & (x,v) & \longmapsto & x+v \end{array} .
\end{displaymath}

Then, it sufices to proof the following lemma:

\begin{lema}
The point $p = x + v$ is a critical value of $E$ if, and only if, the matrix

\begin{displaymath}
\frac{\partial^2 f_p}{\partial u_i \partial u_j} = 2 \left( \frac{\partial x}{\partial u_i} \cdot \frac{\partial x}{\partial u_j} - v \cdot \frac{\partial^2 x}{\partial u_i \partial u_j} \right)
\end{displaymath}

is not invertible.
\end{lema}

\begin{proof}
Consider, for each point of the local chart $(u_1,...,u_d)$, a orthonormal basis of $(\straightT_xM)^{\bot}$ by the $N-d$ vectors $v_1,...,v_{N-d}$. Then, we have a local parametrization for $N$, given by the map

\begin{displaymath}
(u_1,...,u_d,t_1,...,t_{N-d}) \longmapsto \left( x(u_1,...,u_d), \sum_{i=1}^{N-d} t_i v_i (u_1,...,u_d) \right) .
\end{displaymath}

In these coordinates, the partial derivatives of $E$ are

\begin{displaymath}
\left\{ \begin{array}{l} \frac{\partial E}{\partial u_i} = \frac{\partial x}{\partial u_i} + \sum_{k=1}^{N-d} t_k \frac{\partial v_k}{\partial u_i} \\ \frac{\partial E}{\partial t_j} = v_j \end{array} \right. .
\end{displaymath}

If we compute the inner products of these $N$ vectors with the $N$ independent vectors $\frac{\partial x}{\partial u_1} , ..., \frac{\partial x}{\partial u_d}, v_1, ..., v_{N-d}$, we get a square matrix that has the same rank as the Jacobian of $E$, and this matrix has the form

\begin{displaymath}
\begin{pmatrix} \left( \displaystyle\frac{\partial x}{\partial u_i} \cdot \frac{\partial x}{\partial u_j} + \sum_k t_k \frac{\partial v_k}{\partial u_i} \cdot \frac{\partial x}{\partial u_j} \right) & \left( \displaystyle\sum_k \frac{\partial v_k}{\partial u_i} \cdot v_l \right) \\ 0 & \text{Id} \end{pmatrix} .
\end{displaymath}

But $v_k$ are orthogonal to $\frac{\partial x}{\partial u_j}$, so

\begin{displaymath}
0 = \frac{\partial}{\partial u_i} \left( v_k \cdot \frac{\partial x}{\partial u_j} \right) = \der{v_k}{u_i} \cdot \der{x}{u_j} + v_k \cdot \frac{\partial^2 x}{\partial u_i \partial u_j} ,
\end{displaymath}

so

\begin{displaymath}
\sum_k t_k \der{v_k}{u_i} \cdot \der{x}{u_j} = - \sum_k t_k v_k \frac{\partial^2 x}{\partial u_i \partial u_j} = v \cdot \frac{\partial^2 x}{\partial u_i \partial u_j} .
\end{displaymath}

And, thus, the lemma is proved.
\end{proof}

So the proposition is proved.
\end{proof}

To answer the second question, about how many Morse functions are there in a given manifold, we have the following proposition:

\begin{prop}
Let $M$ be a smooth manifold, and let $f : M \rightarrow \R$ a smooth function. Let $k$ be a positive integer. Then, $f$ and all its derivatives of order $\leq k$ can be uniformly approximated by Morse functions on every compact subset.
\end{prop}

\begin{proof}
Choose an embedding of $M$ into $\R^N$ such that its first coordinate is the function $f$,

\begin{displaymath}
h(x) = (f(x),h_2(x),...,h_N(x)) .
\end{displaymath}

Choose $c$ a (large) real number. By the proposition \ref{existenceMorse}, for almost all point $(-c+\e_1,\e_2,...,\e_N)$ the function $f_p$ is a Morse function, so the function

$$g_{c,\e}(x) = \frac{f_p(x)-c^2}{2c}$$

is also a Morse function. Moreover, notice that

$$g_{c,\e}(x) = \frac1{2c} ((f(x) + c - \e_1)^2 + (h_2(x)-\e_2)^2 + \cdots + (h_N(x)-\e_N)^2 - c^2) =$$

$$= f(x) + \frac{f(x)^2 + \sum h_i(x)^2}{2c} - \frac{\e_1f(x) + \sum \e_ih_i(x)}{c} + \sum \e_i^2 - \e_1 ,$$

so, as claimed, on any compact subset of $M$, $g_{c,\e}$ tends to $f$ and its first $k$ derivatives as $\e \rightarrow 0$ and $c \rightarrow \infty$.
\end{proof}

Therefore, the Morse functions are actually generic, in the sense that we can approximate any smooth function with Morse functions over compact subsets of $M$.

\subsection{Pseudogradients}

In this subsection, we will focus on how we can connect the points of $\crit(f)$ (for a Morse function) between them. The way to do this, in $\R^n$, would be the gradient of the function, this means,

$$\grad f = \left( \der{f}{x_1},...,\der{f}{x_n} \right) .$$

This vector field can be intrinsecally defined in manifolds using the metric $\langle \cdot , \cdot \rangle$ on a manifold $M$, as the only vector field that satisfies that, for any vector field $Y \in \mathfrak{X}(M)$,

$$\langle \grad f, Y \rangle = \dd f \cdot Y .$$

For our purposes, we do not need to use precisely the gradient of a function, and sometimes it might not satisfy the properties that we need from the flow, so we introduce the more general concept of {\bf pseudogradient}.

\begin{deff}
Let $f$ be a Morse function defined on $M$. Then, a vector field $X \in \mathfrak{X}(M)$ is a {\bf pseudogradient} adapted to $f$ if it satisfies the two conditions

\begin{enumerate}
\item For every $x \in M$ we have $(\dd f)_x \cdot X_x \leq 0$, and the equality hold if and only if $x$ is a critical point for $f$.
\item For each critical point of $f$, $X$ coincides with the negative gradient defined with the canonical metric defined in $\R^n$ on the domain of the Morse chart.
\end{enumerate}
\end{deff}

\begin{rmrk}
If $(M,g)$ is a manifold with a Riemannian metric, then the vector field $- \grad f$ defined using $g$ is a pseudogradient.
\end{rmrk}

As with Morse functions, one might ask if pseudogradients do exist. As before, the answer is affirmative:

\begin{prop}
Let $M$ be a compact and smooth manifold, and $f$ a Morse function on $M$. Then, there exists a pseudogradient field $X$ adapted to $f$.
\end{prop}

\begin{proof}
Let $c_1,...,c_r$ be the critical points of $f$ (there are finitely many because the non-degenerate critical points are isolated and $M$ is compact), and let $(U_1,h_1),...,(U_r,h_r)$ be Morse charts in the neighbourhoods of these points. We assume that the images of these charts, which we will denote $\Omega_1,...,\Omega_r$, are disjoint. We can add more charts to get a finite open cover of $M$, $\{\Omega_j\}_{1 \leq j \leq m}$. Let us assume that each critical point $c_i$ is contained only on its associated open set $\Omega_i$.

For each $\Omega_i$, let $X_i$ be the push-forward of the negative gradient of $\tilde{f}$ in $U_i$ by $h_i$. Also, take a partition of unity associated to $\{\Omega_j\}_{1 \leq j \leq m}$, $\{\varphi_j\}_{1 \leq j \leq m}$. Then, we can define the vector fields

$$\tilde{X}_j(x) = \left\{ \begin{array}{ll} \varphi_j(x) X_j & \text{if } x \in \Omega_j \\ 0 & \text{otherwise} \end{array} \right. ,$$

and

$$X = \sum_{j=1}^m \tilde{X}_j .$$

Then, the two conditions to be a pseudogradient adapted to $f$ are satisfied:

\begin{enumerate}
\item If we compute

$$\\d f_x \cdot X_x = \sum_{j=1}^m \\d f_x \cdot \tilde{X}_{j,x} \leq 0 ,$$

and the inequality is an equality iff $\varphi_j(x) X_j(x) = 0$ for every $j$, so that either $x$ is a critical point, or $\varphi_j(x)=0$ for all $j$, which is impossible since $\{\varphi_j\}_{1\leq j \leq m}$ form a partition of unity.
\item Let $c_i$ be a critical point. By construction, $X$ coincides with the image of the negative gradient with the canonical metric over $U_i \bigcap \left( \bigcup_{i \neq j} U_j\right)$, which, also by construction, contains a neighbourhood of $c_i$.
\end{enumerate}
\end{proof}

Going back to the question of how to connect two critical points, let us define the stable and unstable manifolds:

\begin{deff}
Let $a$ be a critical point of $f$, and $X$ a pseudogradient adapted to $f$. We denote by $\varphi^s$ the flow of $X$. We define the {\bf stable manifold} of $a$ as

$$W^s(a) = \left\{ x \in M \ | \ \lim_{s \rightarrow + \infty} \varphi^s(x) = a \right\}, $$

and its {\bf unstable manifold} as

$$W^u(a) = \left\{ x \in M \ | \ \lim_{s \rightarrow - \infty} \varphi^s(x) = a \right\} .$$
\end{deff}

These are, actually, smooth manifolds, and it can be shown that they are diffeomorphic to open disks. Moreover, we have that

$$\text{dim} W^u(a) = \text{codim} W^s(a) = \text{Ind}(a) .$$

Finally, we want to see that the flow of a pseudogradient does actually connect critical points. To this end, we have the following result:

\begin{prop}
Let $M$ a compact manifold, $f$ a Morse function and $\gamma : \R \rightarrow M$ a trajectory of a pseudogradient field $X$. Then, there are critical points $a$ and $b$ such that

\begin{displaymath}
\lim_{s \rightarrow - \infty} \gamma(s) = a \ , \text{ and } \lim_{s \rightarrow + \infty} \gamma(s) = b .
\end{displaymath}
\end{prop}